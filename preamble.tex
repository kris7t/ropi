\documentclass[a4paper,11pt,twoside]{article}
\renewcommand*{\baselinestretch}{1.04}
\usepackage[margin=2cm,head=16pt,foot=0pt,includeheadfoot,headsep=16pt]{geometry}

\usepackage[T1]{fontenc}
\usepackage[utf8]{inputenc}
\usepackage{ClearSans,lmodern}

\usepackage{paralist}
\pltopsep=0pt
\plitemsep=0pt
\plparsep=0pt

\def\+{+}

\makeatletter
\catcode`\ =12\let\@nl@space= \catcode`\ =10
\newcount\@nl@rlevel
\newcount\@nl@llevel
\@nl@llevel=-1

\def\@nl{%
  \catcode`\ =12
  \global\@nl@rlevel=0
  \futurelet\@nl@store\@nl@%
}
\def\@nl@gobble#1{\futurelet\@nl@store\@nl@}
\def\@nl@enditemize{
  \ifnum\the\@nl@rlevel<\the\@nl@llevel%
    \end{compactitem}%
    \egroup%
    \expandafter\@nl@enditemize%
  \else%
    \ifnum\the\@nl@rlevel=\the\@nl@llevel\else%
       \errmessage{Error: inconsistent identation}
    \fi%
  \fi%
}
\def\@nl@{%
  \ifx\@nl@store\@nl@space%
    \global\advance\@nl@rlevel by 1
    \expandafter\@nl@gobble%
  \else%
    \catcode`\ =10
    \ifx\@nl@store+%
      \ifnum\the\@nl@rlevel>\the\@nl@llevel%
        \bgroup%
        \@nl@llevel=\the\@nl@rlevel
        \begin{compactitem}%
      \fi%
      \@nl@enditemize%
      \item \expandafter\expandafter\expandafter\@gobble%
    \else%
      \ifx\@nl@store\@nl%
        \global\@nl@rlevel=-1\relax\@nl@enditemize\par
      \else\space\fi%
    \fi%
  \fi%
}
\makeatother

\usepackage{fancyhdr}

\def\magyarOptions{defaults=prettiest}
\usepackage[magyar]{babel}

\usepackage{amsmath,wasysym,mathtools,mleftright,relsize,amssymb}

\usepackage[stretch=10]{microtype}

\usepackage{xspace,mdframed,array,needspace}

\fancyhead{}
\fancyfoot{}
\fancyhead[LE,RO]{\thepage}
\fancyhead[LO]{\scshape\nouppercase{\rightmark}}
\fancyhead[RE]{\scshape\nouppercase{\leftmark}}
\pagestyle{fancy}

\newmdenv[nobreak=true,skipabove=3ex,skipbelow=1ex,%
  backgroundcolor=black!15,hidealllines=true]{tetelframe}
\setcounter{secnumdepth}{-1}
\newcounter{tetel}
\newcommand{\tetel}[1]{%
  \refstepcounter{tetel}\markright{\thetetel.~tétel}%
  % Ugly hyperref hack
  \stepcounter{section}\phantomsection%
  \addcontentsline{toc}{subsection}{\thetetel.~tétel}
  \pagebreak[3]%
  \begin{tetelframe}\noindent%
    \textbf{\thetetel.}~#1\end{tetelframe}}
\raggedbottom

\newcommand*{\ra}{\ensuremath{\rightarrow}\xspace}
\newcommand*{\RA}{\ensuremath{\Longrightarrow}\xspace}
\newcommand*{\LA}{\ensuremath{\Longleftarrow}\xspace}
\newcommand*{\app}{\ensuremath{\approx}\xspace}
\newcommand*{\lra}{\ensuremath{\leftrightarrow}\xspace}
\newcommand*{\LRA}{\ensuremath{\Longleftrightarrow}\xspace}

\newcounter{theorem}[tetel]
\renewcommand*{\thetheorem}{\thetetel.\arabic{theorem}}
\newcommand*{\theoremlike}[1]{\refstepcounter{theorem}{\textbf{\thetheorem.~#1:}}\xspace}
\newcommand*{\thm}{\theoremlike{tétel}}
\newcommand*{\example}{\theoremlike{példa}}
\newcommand*{\prob}{\theoremlike{probléma}}
\newcommand*{\alg}{\theoremlike{algoritmus}}
\newcommand*{\dfn}{\theoremlike{definíció}}
\newcommand*{\lemma}{\theoremlike{lemma}}
\newcommand*{\corr}{\theoremlike{következmény}}
\newcommand*{\stmnt}{\theoremlike{állítás}}
\newcommand*{\proof}{\textit{bizonyítás:}\xspace}
\newcommand*{\noproof}{\textit{nem bizonyítjuk}}
\newcommand*{\qed}{\hfill\ensuremath{\blacksquare}}
\newcommand*{\RR}{\mathbb{R}}
\newcommand*{\QQ}{\mathbb{Q}}
\newcommand*{\CR}{\mathcal{R}}
\newcommand*{\ZZ}{\mathbb{Z}}
\newcommand*{\Nat}{\mathbb{N}}
\newcommand*{\GF}{\textrm{GF}_2}
\newcommand*{\MM}{\mathcal{M}}
\newcommand*{\NN}{\mathcal{N}}
\newcommand*{\UU}{\mathcal{U}}
\newcommand*{\FF}{\mathcal{F}}
\newcommand*{\BB}{\mathcal{B}}
\newcommand*{\CC}{\mathcal{C}}
\newcommand*{\T}{\mathrm{T}}
\newcommand*{\IP}{\textit{IP}}
\newcommand*{\LP}{\textit{LP}}
\newcommand*{\DLP}{\textit{DLP}}
\newcommand*{\DIP}{\textit{DIP}}
\newcommand*{\cookprec}{\mathbin{{\prec}_{\textrm{Cook}}}}

\newcommand*{\DataIn}{\textsc{Input}:\xspace}
\newcommand*{\DataOut}{\textsc{Output}:\xspace}

\newcommand{\vertle}{\rotatebox{90}{$\mkern-2mu\le$}}
\newcommand{\vertgt}{\rotatebox{90}{$\mkern-2mu>$}}
\DeclareMathOperator*{\argmax}{arg\,max}
\DeclareMathOperator*{\argmin}{arg\,min}

\usepackage{ocg-p,tikz}
\usetikzlibrary{positioning,calc,shapes.geometric,through}
\let\tikzpictureunpatched\tikzpicture
\def\tikzpicture{\catcode`\^^M=5\tikzpictureunpatched}
\tikzset{
  vertex/.style={shape=circle,fill=black,inner sep=0pt,
    font=\nullfont,minimum width=4pt}
}

\colorlet{forkmeshadow}{black!25}
\definecolor{forkmebg}{HTML}{CC0505}
\definecolor{forkmebg2}{HTML}{A00000}
\definecolor{forkmefg}{HTML}{EEEEEE}
\pgfdeclareverticalshading{forkmeshading}{600pt}{%
  color(0pt)=(white);
  color(5pt)=(forkmeshadow);
  color(5.001pt)=(forkmebg);
  color(16pt)=(forkmebg2);
  color(21pt)=(forkmebg2);
  color(36.999pt)=(forkmebg);
  color(37pt)=(forkmeshadow);
  color(42pt)=(white)}

\title{Rendszeroptimalizálás vizsgatételek (2015/2016.~második
  félév)}
\author{Marussy Kristóf}

\makeatletter
\usepackage[pdftitle={\@title},pdfauthor={\@author},bookmarks,unicode,hidelinks]{hyperref}
\makeatother
\usepackage{bookmark}