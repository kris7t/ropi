
\section{Lineáris programozás}

\tetel{Az optimális hozzárendelés problémája, Egerváry algoritmusa.}

+ \example egy cég számos megrendelést kap különboző ``egyszemélyes'',
  egyforma idő alatt elvégezhető munkák elvégzésére
  + kimutatást készítünk arról, hogy melyik dolgozó melyik munkát
    tudja elvégezni
  + cél a profit maximalizálása (lehető legtöbb munka elvégzése)
+ \alg ``magyar módszer''
  + \DataIn $G = (F, L; E)$ páros gráf
  + \DataOut $M \subseteq G$ egy maximális méretű párosítás
  + induljunk ki egy tetszőleges (pl.~az üres) $M$ párosításból
  + \emph{alternáló út} = párosítatlan $F$-beli csúcsból indul, $\forall$
    második éle az $M$-hez tartozik
    + \emph{javító út} = olyan alternáló út, ami párosítatlan $L$-beli
      pontban ér véget
  + amíg találunk $J$ javítóutat (pl.~szélességi kereséssel) \RA $M
    \gets M - (J \cap M) \cup (J - M)$
+ \thm a ``magyar módszer'' valóban maximális párosítást talál
  $G$-ben
  + \proof $M \coloneqq \text{a párosítás, amit az algoritmus
    megtalált}$
  \begin{align*}
    F_1 &\coloneqq F - M, \text{az $M$ által le nem fedett $F$-beli
          pontok halmaza,}\\
    L_2 &\coloneqq \text{az $F_1$-ből alternáló úton elérhető pontok
          halmaza,} \\
    F_2 &\coloneqq \text{az $L_2$-beli pontok $M$ szerinti párjai,} \\
    L_3 &\coloneqq \text{az $F_1$-ből alternáló úton nem elérhető
          $L$-beli pontok halmaza,} \\
    F_3 &\coloneqq \text{az $L_3$-beli pontok $M$ szerinti párjai,} \\
    L_1 &\coloneqq L - M, \text{az $M$ által le nem fedett $L$-beli pontok
          halmaza}
  \end{align*}
  + vegyük észre, hogy $G$-ben nem vezethet $F_1 \cup F_2$ és $L_1
    \cup L_3$ között él
    \par
    {\centering\begin{tabular}{c|p{6cm}|p{6cm}}
      & \multicolumn{1}{c|}{$F_1$} & \multicolumn{1}{c}{$F_2$} \\\hline
      $L_1$ & $1$ hosszú javító út lenne & $\ge 3$ hosszú javító út lenne \\\hline
      $L_3$ & az $L_3$-beli csúcs $1$ hosszú alternáló úton elérhető lenne, azaz $L_2$-beli
            & az $L_3$-beli csúcs $\ge 3$ hosszú alternáló úton elérhető lenne, azaz $L_2$-beli
    \end{tabular}\par}
  + $F_1 \cup F_2$ $\forall$ szomszédja $L_2$-beli, $L_2 \cup F_3$ egy lefogó ponthalmaz
  + mivel épp $\lvert M \rvert = \lvert L_2 \cup F_3 \rvert$, $M$
    valóban maximális méretű \qed
+ \prob optimális hozzárendelés
  + \DataIn $G = (F, L; E)$ páros gráf, $w\colon E \to \RR$
    súlyfüggvény
  + \DataOut $M \subseteq G$ párosítás úgy, hogy $\sum_{e \in M} w(e)$
    maximális
  + az optimális hozzárendelés megoldható maximális súlyú
    \emph{teljes} párosítás keresésével
    + ha $\lvert F \rvert \ne \lvert L \rvert$, adjunk $G$-hez annyi
      csúcsot, hogy egyenlőek legyenek
    + legyen $G' = (F, L; E')$ teljes páros gráf ($E' = F
      \times L$), $w'(e) \coloneqq \text{\footnotesize$\begin{cases}
        w(e), &\text{ha $e \in E$,} \\
        0, & \text{ha $e \notin E$}
      \end{cases}$}$
    + $G'$ egy $M'$ maximális súlyú teljes párosítasa a $G$ maximális
      súlyú párosítása az $E' - E$ élek elhagyása után
    + $G$ maximális súlyú $M$ párosításához megfelelő $E' - E$ éleket
      hozzávéve $G'$ maximális súlyú teljes párosítását kapjuk
+ \dfn a $c\colon F \cup L \to \RR$ függvény \emph{címkézés} a $G =
  (F, L; E)$ páros gráfra és a $w\colon E \to \RR$ súlyfüggvényre
  nézve, ha $\forall e = \{ x, y \} \in E\colon c(x) + c(y) \ge w(e)$
+ \lemma a $w$ súlyfüggvénnyel súlyozott $G = (F, L; E)$ páros gráf
  tetszőleges $M$ teljes párosítására és tetszőleges $c$
  címkézésére igaz, hogy $\sum_{e \in M} w(e) \le \sum_{v \in F
    \cup L} c(v)$
  + \proof $\displaystyle \sum_{e \in M} w(e) \le \sum_{\mathclap{e = \{f, l\} \in M}}\, c(f) +
    c(l) \overset{\substack{\text{$\forall$ csúcs legfeljebb}\\\text{$1$-szer szerepel $M$-ben}}}{\le} \sum_{f \in F} c(f) + \sum_{l \in L} c(l) = \sum_{\mathclap{v \in F
    \cup L}}\, c(v)$ \qed
  + \dfn a $G$ gráf $e = \{x, y\}$ éle \emph{piros}, ha $c(x) +
    c(y) = w(e)$
  + \corr ha az $M$ teljes párosítás $\forall$ éle piros valamely $c$
    címkézésre nézve, akkor $M$ maximális súlyú teljes párosítás
+ \alg Egerváry algoritmusa
  + \DataIn $G = (F, L; E)$ páros gráf, $w\colon E \to \RR$
  súlyfüggvény
  + \DataOut $M \subseteq G$ \emph{teljes} párosítás úgy, hogy $\sum_{e \in M} w(e)$ maximális
  + 0.~lépés: legyen $M = \emptyset$, $c(v) = \begin{cases}
    \max_{y \in L, \{v, y\} \in E} w({v, y}), & \text{ha $v \in F$,} \\
    0, & \text{ha $v \in L$}
  \end{cases}$
  + 1.~lépés: a javító utas algoritmussal keressünk bővítsük $M$-et
    maximális élszámú párosítássá a piros részgráfban
    + ha $M$ most már teljes \RA STOP, $M$ a keresett párosítás, $c$ a
    keresett címkézés, $w(M) = c(M)$
  + 2.~lépés: $\delta \coloneqq \min \{ c(x) + c(y) - w(\{x, y\}) \mid
    \{x, y\} \in E, x \in F_1 \cup F_2, y \in L_1 \cup L_3 \}$
    + állítsuk elő a $c(v) \gets \text{\footnotesize$\begin{cases}
      c(v) - \delta, & \text{ha $v \in F_1 \cup F_2$,} \\
      c(v) + \delta, & \text{ha $v \in L_2$,} \\
      c(v), & \text{ha $v \in F_3 \cup L_1 \cup L_3$}
    \end{cases}$}$ új címkézést, majd GOTO 1.
+ \thm Egerváry algortimusa $O(n^2 e)$ lépésben maximális súlyú teljes
  párosítást állít elő
  + \proof a 0.~lépésben megadott $c$ valóban címkézés
  + a 2.~lépésben $\delta$ kiszámításához valóban van él $F_1 \cup
  F_2$ és $L_1 \cup L_3$ között
    + ha nem lenne, $N(F_1 \cup F_2) = L_2$
    + $\lvert F_1 \cup F_2 \rvert > \lvert F_2 \rvert = \lvert L_2
    \rvert$ miatt ekkor nem teljesül a Hall-feltétel \RA $\nexists$
    teljes párosítás!
  + a 2.~lépés után is címkézés marad $c$
    + csak az $F_1 \cup F_2$ és $L_1 \cup L_3$ között vezető élekre
    csökken $c(x) + c(y)$
    \par
    {\centering
      \begin{tabular}{c|c|c|c}
        & $F_1$ & $F_2$ & $F_3$ \\\hline
        $L_1$ & $-\delta$ (nem lehet piros) & $-\delta$ (nem lehet piros) & $0$ \\\hline
        $L_2$ & $-\delta + \delta$ & $-\delta + \delta$ & $+\delta$
                                                          (piros eltűnhet) \\\hline
        $L_3$ & $-\delta$ (nem lehet piros) & $-\delta$ (nem lehet piros) & $0$
      \end{tabular}\par}
    + $\delta$ definíciója garantálja, hogy továbbra is $c(x) + c(y)
      \ge w(\{x, y\})$
  + $M$ élei a 2.~lépés után is pirosak
    + csak $x \in F_3$, $y \in L_2$ élek színeződhetnek vissza (itt nőtt
      $c(x) + c(y)$)
    + ezek nem lehetnek $M$ élei, mert $F_3$ párja $L_3$, $L_2$ párja
    $F_2$
    + továbbra is van $F_1$-ből piros alternáló út $L_2$ csúcsaiba
  + egy iterációban vagy $M$, vagy $L_2$ elemszáma nő
   + $O(n)$ lépés után $M$ elemszáma mindenképp nő, mert ekkorra már $L_2$
     lefedné $L$-t
   + $O(n^2)$ iterációban $M$ maximális párosítás lesz
   + ezért $O(n^2 e)$ időben az algoritmus véget ér \qed

\tetel{A lineáris programozás alapfeladata, kétváltozós feladat
  grafikus megoldása. Lineáris egyenlőtlenségrendszer megoldása
  Fourier`--Motzkin eliminációval.}

+ legyen $A \in \RR^{m \times n}$, $x \in \RR^n$, $b \in \RR^m$, $c \in
  \RR^n$ \RA \emph{lineáris program}
+ \dfn \emph{linerási programozás alapfeladata}: $\min_{x} \{ cx : Ax
  \le b \}$
+ kétdimenziós feladat megoldása
  + az $p_1 x_1 + p_2 x_2 \ge k$ alakú feltétel $p_1 x_1 + p_2
  x_2 = k$ egyenese két félsíkra bontja a síkot
    + a $\ge$ jelnek megfelelő félsíkok metszete adja megengedett
      megoldások tartományát
  + ha $c = (q_1, q_2)$ \RA $q_1 x_1 + q_2 x_2 = 0$-val párhuzamos
  egyeneseket húzunk
    + minden egyenesre kiszámítjuk a célfüggvény értékét
    + a maximumhely a legnagyobb egyenes és a félsíkok metszetének
    közös pontja
  + általánosítás: \emph{hipersíkok} által határolt \emph{poliéder}
+ ekvivalens átalakítások
  + egyenlőtlenség megszorzása \emph{pozitív} számmal
  + két egyenlőtlenség összegének hozzávétele az egyenlőtlenségrendszerhez
  + \emph{nem} ekvivalens: szorzás negatív számmal
+ \alg Fourier`--Motzkin elimináció
  + $n$ változós lineáris program visszavezetése egy $n - 1$ változós
    $A^*$, $b^*$ lineáris programra
  + végül $1$ változós lineáris program megoldása
  + $(A|b)$ bővített együttható mátrix
    + szorozzuk pozitív számokkal a
      sorokat, hogy az $1$.~oszlopban csak $-1, 0, 1$ legyen
  + legyen $I$, $J$, $K$ rendre az $1$-gyel, $-1$-gyel és $0$-val
    kezdődő sorok indexeinek halmaza
  + $\RR^{\lvert K \rvert \times n} \ni (A_0|b_0) =$ $(A|b)$ $0$-val
    kezdődő sorai az első oszlop elhagyásával
  + $Ax \le b$ egy megoldása $x = (\lambda, \bar{x})$ alakú \RA $A_0
    \bar{x} \le b_0$ \RA mikor van megfelelő $\lambda$?
  + ha $J = \emptyset$ (nincs $-1$-gyel kezdődő sor)
    + $\forall i \in I : \lambda + \bar{a}_i \bar{x} \le b_j$ \RA
      $\lambda \le \min_{i \in I} b_i - \bar{a}_i \bar{x}$, ami mindig
      kielégíthető
    + $A^*, b^* =$ az $\bar{A}$-ból és $b$-ből az $i \in I$ indexű sorok
      elhagyásával kapott rendszer
  + ha $I = \emptyset$ (nincs $1$-gyel kezdődő sor)
    + $\forall j \in J : -\lambda + \bar{a}_j \bar{x} \le b_j$ \RA
      $\lambda \ge \max_{j \in J} \bar{a}_j \bar{x} - b_j$, ami mindig
      kielégíthető
    + $A^*, b^* =$ az $\bar{A}$-ból és $b$-ből az $j \in J$ indexű sorok
      elhagyásával kapott rendszer
  + ha $I \ne \emptyset$ és $J \ne \emptyset$ \RA
    $\forall i \in I : \lambda + \bar{a}_i \bar{x} \le b_j$ és
    $\forall j \in J : -\lambda + \bar{a}_j \bar{x} \le b_j$
    + $\forall i \in I, j \in J: \bar{a}_j \bar{x} - b_j \le \lambda
      \le \bar{b}_i - \bar{a}_i \bar{x}$ \RA $\forall i \in I, j \in J:
      (\bar{a}_i + \bar{a}_j) \bar{x} \le b_i + b_j$
    + $A^*, b^* =$ az $\bar{A}$ és $b$-ből $i \in I$ és $j \in J$
      indexű sorait az összes lehetséges módon összeadjuk, a $k \in K$
      indexű sorokat hozzávesszük és a csupa $0$ első oszlopot
      elhagyjuk
  + egyváltozós rendszer megoldása
    + ha $\exists k \in K : b_k < 0$ \RA a rendszer nem megoldható
    + ha $\max_{j \in J} -b_j > \min_{i \in I} b_i$ \RA a rendszer nem
      megoldható
    + egyébként \RA $x_1 \in \bigl[\max_{j \in J} -b_j, \min_{i \in I}
      b_i\bigr]$ megoldás \qed

\tetel{Farkas-lemma (két alakban). A lineáris program célfüggvénye
  felülről korlátosságának feltételei.}

+ \lemma Farkas-lemma, 1.~alak
  + tetszőleges $A$ és $b$ esetén az alábbi két rendszer közül
    pontosan egynek van megoldása:\\
    (1)~$Ax \le b$\qquad(2)~$yA = 0$, $y \ge 0$, $yb < 0$
+ \proof (1)~megoldható \RA (2)~nem megoldható
  + $0 = 0x = (yA) x = y (Ax) \le yb < 0$, ellentmondás
+ (1)~nem megoldható \RA (2)~megoldható
  + alkalmazzuk a Fourier`--Motzkin eliminációt az
    (1)~egyenletrendszerre
  + a feltevés szerint a kapott egyváltozós $(A^*|b^*)$ rendszer nem
    megoldható
    + ha van $(0|\beta)$, $\beta < 0$ sor
      + $\exists$ $(A|b)$ sorainak olyan nemnegatív $y$ együtthatós
      lin.~kombinációja, hogy\\
        $y (A|b) = (0, 0, \ldots, 0|\beta)$ \RA $y$ kielégíti (2)-t
    + ha van $(-1, \beta_j)$ és $(1,\beta_i)$ sor úgy, hogy $-\beta_j >
    \beta_i$ \RA $\beta = \beta_i + \beta_j < 0$
      + $\exists y_j \ge 0: y_j (A|b) = (0, 0, \ldots, 0, -1 | \beta_j)$
        és $\exists y_i \ge 0: y_i (A|b) = (0, 0, \ldots, 0, 1 | \beta_i)$
      + ekkor $y = y_i + y_j \ge 0$, $y (A|b)
      = (0, 0, \ldots, 0 | \beta)$ \RA $y$ kielégíti (2)-t \qed
+ \lemma Farkas-lemma, 2.~alak
  + tetszőleges $A$ és $b$ esetén az alábbi két rendszer közül
    pontosan egynek van megoldása:\\
    (1)~$Ax = b$, $x \ge 0$\qquad(2)~$yA \ge 0$, $yb < 0$
  + \proof (1)~megoldható \RA (2)~nem megoldható
    + $0 \le (yA)x = y(Ax) = yb < 0$, ellentmondás
  + (2)~nem megoldható \RA (1)~megoldható
    + vizsgáljuk az ekvivalens $y(-A) \le 0$, $yb = -1 < 0$  rendszert
    + tömör alakban $(-A|b|-b)^\T y \le (0, 0, \ldots, 0, -1, 1)^\T$
    + a lemma 1.~alakja szerint $\exists x: x (-A|b|-b)^\T = 0, x
      \ge 0, x (0, 0, \ldots, 0, -1, 1)^\T < 0$
    + legyen $x = (\bar{x} | \lambda, \mu)$ \RA $-A\bar{x} + (\lambda -
      \mu)b = 0$, $\bar{x} \ge 0$, $\lambda, \mu \ge 0$, $-\lambda +
      \mu < 0$
    + ekkor $A\frac{\bar{x}}{\lambda - \mu} = b$ és $\lambda - \mu > 0$
      miatt $\frac{\bar{x}}{\lambda - \mu} \ge 0$ \RA%
      $\frac{\bar{x}}{\lambda - \mu}$ kielégíti (1)-et \qed
+ \thm \label{thm:linprog:farkas:3kalicka}ha $Ax \le b$ megoldható,
  $c$ tetszőleges \RA az alábbi állítások ekvivalensek
  + (1)~az $Ax \le b$ megoldáshalmazán $cx$ felülről korlátos
  + (2)~nincs megoldása az $Az \le 0$, $cz > 0$ rendszernek
  + (3)~van megoldása az $yA = c$, $y \ge 0$ rendszernek
  + \proof (1) \RA (2), legyen $x_0$ (1) egy megoldása és
    indir.~tfh.~$z$ (2) egy megoldása
    + ekkor $A(x_0 + \lambda z) \le b$, de $\lambda > 0$ esetén
      $c(x_0 + \lambda z) = c x_0 + \lambda c z$ tetszőlegesen nagy
    + $cx$ nem felülről korlátos \RA ellentmondás
  + (2) \RA (3), tekintsük a (nem megoldható) $z (-A)^\T \ge 0$, $z
    (-c) < 0$ rendszert
    + a Farkas-lemma 2.~alakja szerint $\exists y: (-A)^\T y = -c, y \ge 0$
    + $(-A)^\T y = -c$ \RA $A^\T y = y A = c$ \RA ez az $y$ épp
      kielégíti a (3)-as rendszert
  + (3) \RA (1), legyen $y$ a (3)-as rendszer egy megoldása
    + $cx = (yA)x = y (Ax) \overset{y \ge 0, Ax \le b}{\le} yb$ \RA%
    $yb$ a $cx$ egy felső korlátja \qed

\tetel{A lineáris programozás dualitástétele (két alakban). A lineáris
  programozás alapfeladatának bonyolultsága (biz.~nélkül).}

+ \thm \label{thm:linprog:dual:dual}a lineáris programozás dualitástétele
  + ha $\max \{ cx : Ax \le b \}$ (primál) program megoldható és
    felülről korlátos, akkor
    + (1)~$\min \{ yb : yA = c, y \ge 0 \}$ (duális) program megoldható és
      alulról korlátos
    + (2)~a primál programnak $\exists$ maximuma és a duális programnak
      $\exists$ minimuma
    + (3)~$\max \{ cx : Ax \le b \} = \min \{ yb : yA = c, y \ge 0 \}$
+ \lemma \label{lem:linprog:dual:t}legyen $A x \le b$ megoldható, $t
  \in \RR$, de $A x \le b$, $cx \ge t$ nem megoldható
  + ekkor a $yA = c$, $y \ge 0$, $yb < t$ rendszer megoldható
  + \proof alkalmazzuk a Farkas-lemmát a $(A|-c) x \le (b|-t)$-re, $y
    \coloneqq (\bar{y}|\lambda)$
    + $\bar{y} A - \lambda c = 0$, $\bar{y} \ge 0$, $\lambda \ge 0$,
      $\bar{y} b - \lambda t < 0$
    + ha $\lambda = 0$ \RA $0 = 0 x = (\bar{y} A) x = \bar{y} (Ax)
      \overset{\bar{y} \ge 0, Ax \le b}{\le} \bar{y} b < 0$ \RA%
      lehetetlen
    + ezek szerint $y = \frac{\bar{y}}{\lambda}$ létezik, $y A = c$,
      $y \ge 0$, $yb < t$ \qed
+ \emph{\aref{thm:linprog:dual:dual}.~tétel bizonyítása:}
  + (1): már beláttuk a ``3 kalickás tétel''
  (\ref{thm:linprog:farkas:3kalicka}.~tétel) (1)~\LRA~(3) eseténél
    + $cx \le yb$ miatt $cx$ felülről, $yb$ alulról korlátos
  + (2), primál állítás: legyen $t = \sup \{ cx : Ax \le b \}$
    + indir.~tfh.~$\nexists x : Ax \le b, (t \ge )\,cx \ge t$ \RA alkalmazzuk
    \aref{lem:linprog:dual:t}.~lemmát
    + $\exists y : yA = c, y \ge 0, yb < t$ \RA $y$ a duális egy
      megoldása
    +  $t = \sup \{ cx : Ax \le b \} \le yb < t$ \RA ellentmondás \RA $\exists$ primál
      megoldás, hogy $cx \ge t$
    + így a szuprémum egyben maximum kell legyen
  + (2), duális állítás: $\min \{ yb : yA = c, y \ge 0 \} = - \max \{
    -by : A^\T y \le c, -A^\T y \le -c, -y \le 0 \}$
    + alkalmazzuk a primál állítást a duálisra, mint primál feladatra
  + (3): indir.~tfh.~$\exists t : \max \{ cx : Ax \le b \} < t < \min
    \{ yb : yA = c, y \ge 0 \}$
    + a \aref{lem:linprog:dual:t}.~lemma szerint $\exists y : yA = c,
      y \ge 0, yb < t$
    + $t < \min \{ yb : yA = c, y \ge 0 \} < yb < t$ \RA ellentmondás
    \qed
+ \thm \label{thm:linprog:dual:dual}a lineáris programozás
  dualitástétele, ekvivalens alak
  + ha $\max \{ cx : Ax \le b, x \ge 0 \}$ (primál) program megoldható és
    felülről korlátos, akkor
    + (1)~$\min \{ yb : yA \ge c, y \ge 0 \}$ (duális) program megoldható és
      alulról korlátos
    + (2)~a primál programnak $\exists$ maximuma és a duális programnak
      $\exists$ minimuma
    + (3)~$\max \{ cx : Ax \le b, x \ge 0 \} = \min \{ yb : yA \ge c,
      y \ge 0 \}$
  + \proof $\max \{ cx : (A | {-I}) x \le (b | 0) \}$ duálisa $\min \{ y
    (b | 0) : y (A | {-I}) = 0, y \ge 0 \}$
    + legyen $y = (y_1 | y_2)$ \RA $\min \{ y_1 b : y_1 A - y_2 = 0,
      y_1 \ge 0, y_2 \ge 0\}$
    + $y_1 A = y_2 \ge 0$ \RA a duális valóban
      $\min \{ y_1 b : y_1 A \ge 0, y_1 \ge 0 \}$ alakú \qed
+ lineáris programozás bonyolultsága
  + döntési probléma
    + \DataIn $A$ mátrix, $b$ és $c$ vektorok, $t \in \RR$
    + \DataOut van-e olyan $x$ vektor, melyre $Ax \le b$, $cx \ge t$
  + NP-beli \ra tanú a feltételeket kielégítő $x$
  + co-NP-beli \ra a dualitástétel szerint tanú egy $yA = c$, $y \ge
    0$, $(cx \le )\,yb < t$ vektor
  + 1947 (Dantzig): \emph{szimplex-módszer}
    + nem polinomiális futásidejű, de a gyakorlatban gyors
  + 1979 (Hacsijan): \emph{ellipszoid-módszer}
    + polinomiális futásidejű \RA bizonyítja, hogy a feladat P-ben van
    + gyakorlatban a szimplex sokkal hatékonyabb
  + 1984 (\emph{Karmarkar}): polinomiális és a gyakorlatban is használható módszer

\tetel{Egészértékű programozás: a feladat bonyolultsága, korlátozás és
  szétválasztás (Branch and Bound). Totálisan unimoduláris mátrix
  fogalma, példák. Egészértékű programozás totálisan unimoduláris
  együtthatómátrixszal (biz.~nélkül).}

+ \dfn \emph{egészértékű programozás alapfeladata:} $\max_{x} \{ cx : Ax \le b, x
  \text{ egész} \}$ (IP)
  + duálisa $\min_{y} \{ yb : yA = c, y \ge 0, y \text{ egész} \}$ (DIP)
  + $\max_{\IP} \le \max_{\LP} = \min_{\DLP} \le \max_{\DIP}$ \RA állhat
    $<$ is, nincs általános dualitástétel
+ egészértékű programozás bonyolultsága
  + döntési probléma
    + \DataIn $A$ mátrix, $b$ és $c$ vektorok, $t \in \RR$
    + \DataOut van-e olyan $x$ egészértékű vektor, melyre $Ax \le b$, $cx \ge t$
  + NP-beli \ra tanú a feltételeket kielégítő $x$
  + dualitástétel hiányában a co-NP beliség nem látható be
  + \thm az egészértékű lineáris programozás NP-teljes
    + \proof azt már láttuk, hogy IP $\in$ NP
    + adunk egy MAXFTLN $\prec$ IP Karp-redukciót (MAXFTLN NP-teljes)
    + a $G = (V, E)$ gráf $\forall$ $v_i$ csúcsához vegyünk fel egy
      $x_i \in \ZZ$, $-x_i \le 0$, $x_i \le 1$ változót
      + ha $v_i \in F$ a független ponthalmaz eleme, $x_i = 1$,
        egyébként $x_i = 0$
    + $\forall e = \{ v_i, v_j \} \in E$ élre vegyük még fel az $x_i +
      x_j \le 1$ feltételt
    + a célfüggvény $\sum_{v_i \in V} x_i = \lvert F \rvert$ \RA $c =
      (1, 1, \ldots, 1)$
    + $x$ maximumhely \LRA $F \subseteq E$ a $G$ maximális független
      ponthalmaza
      + $F$ független, mert $x_i + x_j \le 1$ \RA $\{ v_i,
        v_j \} \in E$ legfeljebb egyik vége lehet $F$-beli
      + ha $F' \subseteq E$ független, $\lvert F' \rvert > \lvert F
        \rvert$ \RA $x'$ megoldás, $cx' > cx$
      + $x$ nem lehet maximumhely \RA ellentmondás \qed
    + hasonló 3-SAT $\prec$ IP redukciót is lehet adni (0-1
      változók, termek \ra feltételek)
+ \alg Branch and Bound $\max \{ cx : Ax \le b, f \le x \le g; f, g, x
  \text{ egész} \}$ problémára
  + (IP) feladat szétvágása (IP)$'$ és (IP)$''$ feladatokra
    + választunk egy $x_j$ \emph{elágazási változót} és $f_j \le t <
      g_j$ közbülső értéket
    + az új problémák (IP)$'$: (IP) $\cup\;g'_j \coloneqq t$ és
      (IP)$''$: (IP) $\cup\;f''_j \coloneqq t$
  + részproblémák $\mathcal{L} = \{ \text{(IP$^{(i)}$)} = (f^{(i)},
    g^{(i)}, w^{(i)}) : i = 1, 2, \ldots\}$ listáját tartjuk karban,
    ahol $\max \text{(IP$^{(i)}$)} \le w^{(i)}$
  + $x^*$ az eddig megtalált legjobb megoldás, $z^* = c x^*$ a hozzá
    tartozó függvényérték
  + 0.~lépés: $\mathcal{L} \gets \{ (f, g, \infty) \}$, $z^* \gets
    -\infty$, $x^*$ nem definiált
  + 1.~lépés: ha $\mathcal{L} = \emptyset$ \RA STOP, egyébként
    válasszunk egy (IP$^{(i)}$)-t és töröljük $\mathcal{L}$-ből
  + 2.~lépés: ha $w^{(i)} \le z^*$ \RA nem lehet jobb megoldás, mint
    az eddigi, GOTO 1.
  + 3.~lépés: oldjuk meg az (IP$^{(i)}$)-nek megfelelő (LP$^{(i)}$)
    relaxált LP feladatot
    + ha $\nexists$ megoldás \RA GOTO 1., egyébként legyen
      $x^{(i)}$ a megoldás és $c x^{(i)} = z^{(i)}$
  + 4.~lépés
    + (4\rlap{a}\phantom{b})~ha $z^{(i)} \le z^*$ \RA GOTO 1.,
      $f^{(i)} \le x \le g^{(i)}$-ben már nincs jobb megoldás
    + (4b)~ha $z^{(i)} > z^*$ és $x^{(i)}$ egész vektor \RA $x^* \gets
      x^{(i)}$, $z^* \gets z^{(i)}$, GOTO 1.
    + (4\rlap{c}\phantom{b})~ha $z^{(i)} > z^*$, de $x^{(i)}$ nem
      egész vektor
     + vágjuk két részre az (IP$^{(i)}$) problémát valamely $x_j$
       elágazási változó mentén
     + $\mathcal{L} \gets \mathcal{L} \cup \{(f^{(i)\prime},
       g^{(i)\prime}, z^{(i)}), (f^{(i)\prime\prime},
       g^{(i)\prime\prime}, z^{(i)})\}$, GOTO 1.
  + \thm a B\&B véges sok lépésben leáll és megtalálja (IP) optimumát
    + \proof $f$ és $g$ egész \RA véges sok részprobléma van \RA véges
      sok lépés
    + indir.~tfh.~az eljárás leállt, de $z^* < z_0$, ahol $z_0 = \max
      \text{(IP)}$
    + az algoritmus futása közben mindig volt olyan (IP$^{(i)}$) $\in
      \mathcal{L}$, hogy $z_0 = \max \text{(IP$^{(i)}$)}$
      + kezdetben ez maga (IP)
      + $z_0 \le w^{(i)} \le z^* < z_0$ ellentmondás \RA (IP$^{(i)}$)
        mindig eljut a 4.~lépésig
      + (4\rlap{a}\phantom{b})~nem teljesülhet, mert $z_0 \le z^{(i)}
        \le z^* < z_0$ ellentmondás
      + (4b) után $z^* = z^{(i)} = z_0$ lesz \RA $z_0 = z^* < z_0$
        ellentmondás
      + (4\rlap{c}\phantom{b})-ben (IP$^{(i)}$) vágása után $z_0 = \max
        \text{(IP$^{(i)}$)}{}'$ vagy $z_0 = \max \text{(IP$^{(i)}$)}{}''$
    + $\mathcal{L}$ sosem lesz üres \RA az algoritmus nem áll le \RA
      ellentmondás \qed
  + \emph{branch and bound fa:} (IP) a gyökér, (IP$^{(i)}$) gyerekei
    (IP$^{(i)}$)$'$ és (IP$^{(i)}$)$''$
  + heurisztika (IP$^{(i)}$) választására
    + LIFO, ha az (IP$^{(i)}$) a 3.~lépésben utoljára vizsgált probléma fia
      + inkrementális megoldás \emph{duál szimplex módszerrel}
    + egyébként válasszuk azt a problémát, amire $w^{(i)}$ maximális
  + heurisztika az $x_j$ elágazási változó és $t$ választására
    + $x_j$ az változó, aminek az $\bigl\{ x^{(i)}_j \bigr\}$ törtrésze
      legközelebb van $\frac{1}{2}$-hez, $t \gets \bigl\lfloor
      x^{(i)}_j \bigr\rfloor$
  + a gyakorlatban még a heurisztikákkal sem mindig alkalmas nagy
    problémákhoz
+ \dfn $B \in \RR^{k \times k}$ az $A$ mátrix \emph{négyzetes
  részmátrixa}, ha $A$ tetszőleges $k$ sorának és $k$ oszlopának
  kereszteződései határozzák meg
+ \dfn $A$ \emph{totálisan unimoduláris}, ha $\forall$ négyzetes
  részmátrixára $\det = 0, 1 \text{ vagy } {-1}$
  + \corr ha $A$ TU \RA $A$ minden eleme $0$, $1$ vagy $-1$ ($1 \times
    1$-es részmátrix)
+ \lemma \label{lem:linprog:ip:tu}egy mátrix totálisan unimoduláris marad, ha
  + (1)~egy sorát vagy oszlopát $(-1)$-gyel szorozzuk
  + (2)~egységvektort veszünk hozzá új sorként vagy oszlopként
  + (3)~egyik sorát (ill.~oszlopát) új sorként (ill.~oszlopként) új
    példányban hozzávesszük
  + (4)~transzponáljuk
  + \proof csak azoknak a részmátrixoknak változik a $\det$-a,
    amiket a módosítás érint
    + (1)~sor vagy oszlop $(-1)$-gyel szorzása \RA $\det$ $(-1)$-gyel
      szorzása \RA $0$, $1$, $-1$
    + (2)~alkalmazzuk a kifejtési tételt az új sor/oszlop szerint
      + csak egy nemnulla együtthatójú aldetermináns van
      + az új részmátrix determinánsa megegyezik egy régébbiével \RA%
        $0$, $1$ vagy $-1$
    + (3)~két azonos sor/oszlop \RA $\det = 0$
    + (4)~$\det B^\T = \det B$ \RA a determináns nem változik \qed
+ \example \label{ex:linprog:ip:directed}$\vec{G} = (V, \vec{E})$ irányított gráf $A \in \RR^{\lvert
  V \rvert \times \lvert E \rvert}$ illeszkedési
  mátrixsza TU
  + \proof a $B \in \RR^{k \times k}$ részmátrixra $k$ szerinti teljes
    indukcióval
    + $k = 1$-re nyilván $\det B = 0, 1$ vagy $-1$, mert csak ez lehet
      a mátrix eleme
    + ha $B$-ben van csupa $0$ oszlop \RA $\det B = 0$
    + ha $B$-ben van egyetlen $1$-est vagy $-1$-est tartalmazó oszlop
      + fejtsük ki a determinánst eszerint az oszlop szerint
      + az nem $0$ együtthatós $B' \in \RR^{(k - 1) \times (k
        - 1)}$ aldet.-ra $\det B'$ jó az indukció  szerint
      + a kifejtési tétel szerint $\det B = \pm \det B'$
    + ha $B$ $\forall$ oszlopában van $1$ és $-1$ is \RA $(1, 1,
      \ldots, 1) B = 0$ \RA $\det B = 0$ \qed
+ \example \label{ex:linprog:ip:bipartite}$G = (F, L; E)$ páros irányítatlan gráf $A$ illeszkedési
  mátrixsza TU
  + \proof irányítsuk $G$ éleit $F \to L$ irányba \RA $\vec{G}$
    + a $\vec{G}$ irányított gráf $\vec{A}$ illeszkedési mátrixsza TU
      \aref{ex:linprog:ip:directed}.~példa szerint
    + szorozzuk meg az $\vec{A}$ mátrix $L$-hez tartozó sorait $(-1)$-gyel
    + az így kapott mátrix épp $A$, és \aref{lem:linprog:ip:tu}.~lemma
      miatt szintén TU \qed
+ \thm \label{ex:linprog:ip:iplp}ha $A$ TU mátrix, $b$ egész vektor, $c$ valós vektor
  + és $\max \{ cx : Ax \le b \}$ (LP) megoldható és a maximuma véges
  + \RA $\max \{ cx : Ax \le b, x \text{ egész} \}$ (IP) megoldható és
    a maximuma véges
  + és $\max \{ cx : Ax \le b \} = A \max \{ cx : Ax \le b, x \text{
    egész} \}$
  + \noproof

\tetel{A lineáris és egészértékű programozás alkalmazása páros
  gráfokra és intervallumrendszerekre: Egerváry tétele,
  intervallumrendszerek egyenletes színezése.}

+ \thm (Egerváry Jenő tétele) legyen $G = (F, L; E)$ páros gráf, $w\colon E
  \to \RR$ súlyfüggvény \RA a maximális összsúlyú párosítás összsúlya
  $\min \sum_{v \in F \cup L} c(v)$, ahol a minimum
  a nemnegatív $c\colon F \cup L \to \RR_{\ge 0}$ függvényeken értendő, melyekre
  $\forall e = \{x, y\} \in E: c(x) + c(y) \ge w(e)$
  + \proof legyen $G$ illeszkedési mátrixsza $B \in
    \RR^{\lvert F \cup L \rvert \times \lvert E \rvert}$
  + a $B x \le (1, 1, \ldots, 1)^\T$, $x \ge 0$, $x$ egész rendszer
    minden megoldása $0$-$1$ értékű
    + az $1$ komponensek $G$ egy független élhalmazát (párosítását)
      határozzák meg
  + $\max \{ wx : Bx \le (1, 1, \ldots, 1)^\T, x \ge 0, x \text{
    egyész} \}$ megoldása maximális súlyú párosítás
  + $B$ TU mátrix (\ref{ex:linprog:ip:bipartite}.~példa) \RA
    $\max_{\IP} = \max_{\LP}$ (\ref{ex:linprog:ip:iplp}.~tétel)
  + $\max \{ wx : Bx \le (1, 1, \ldots, 1)^\T, x \ge 0 \} = \min \{ y
    (1, 1, \ldots, 1)^\T : yB \ge w, y \ge 0 \}$
  + az $y$ duális megoldás $\forall$ csúcshoz egy $c(v_i) = y_i$
    címkét rendel
    + $y B \ge w$ \RA $\forall \{x, y\} \in E: c(x) + c(y) = w(\{x,
    y\})$ \qed
+ \dfn $G = (\mathcal{I}, E)$ \emph{intervallumgráf}, ha $\mathcal{I}
  = \{I_1, I_2, \ldots, I_m \}$ intervallumok rendszere és $\{ I_i, I_j
  \} \in E$ \LRA $I_i \cap I_j \ne \emptyset$
  + ált.~megsz. nélkült feltehető, hogy $\forall I \in \mathcal{I}$ az $[1, n]$
  egész végpontú, zárt részintervalluma
+ $A(\mathcal{I}) \in \RR^{n \times m}$, ahol $\mathcal{I}$ $[1, n]$ egész
  végpontú intervallumrendszer, $a_{i, j} = \text{\footnotesize$\begin{cases}
      1, &\text{ha $i \in I_j$,} \\
      0, &\text{ha $i \notin I_j$}
    \end{cases}$}$
  + \lemma az így definiált $A(\mathcal{I})$ mátrix TU
  + \proof teljes indukcióval $A$ egyeseinek száma szerint
    + ha $A$-ban $0$ db egyes van \RA $\det A = 0$
    + $\forall$ oszlopban egy darabig $0$-k, utána $1$-esek, majd
      megint $0$-k vannak
    + ha $\exists$ két oszlop, ahol ugyanott van a legfelső egyes
      + a több egyes tartalmazóból a másikat kivonva csökken az
        egyesek száma
      + ez nem változtatja a $\det$-t \RA ind.~feltevés szerint
        $\det = 0, 1$ vagy $-1$
    + ha $\exists$ csupa $0$ oszlop \RA $\det = 0$
    + egyébként minden oszlopban máshol van a legfelső egyes
      + sor- és oszlopcserékkel alsó háromszögmátrixszá alakítható \RA%
      $\det = \pm 1$ \qed
+ \thm \label{thm:liprog:app:interval}az $[1, n]$ egész végpontú, zárt
  $I_1, I_2, \ldots, I_m$
  részintevallumai $\forall k \in \ZZ^{+}$-re megszínezhetőek $k$
  színnel úgy, hogy ha az $i$-t $d_i$ db intervallum tartalmazza,
  akkor ezek között minden színből vagy $\bigl\lfloor \frac{d_i}{k}
  \bigr\rfloor$ vagy $\bigl\lceil \frac{d_i}{k} \bigr\rceil$ van
  + \proof válasszunk ki néhány intervallumot úgy, hogy $\forall i$-re
    $\bigl\lfloor \frac{d_i}{k} \bigr\rfloor$ vagy $\bigl\lceil
    \frac{d_i}{k} \bigr\rceil$ kiválasztott intervallum tartalmazza
    $i$-t
    + ezt az eljárást ismételhetjük $k - 1$-re, $k - 1$-re, \ldots, $1$-re
    + így épp $k$ színnel színezhetőek az intervallumok a megadott
      feltételnek megfelelően
  + legyen $A = A(\mathcal{I})$,
    $\bigl\lfloor \frac{d}{k} \bigr\rfloor$ $i$.~komponense
    $\bigl\lfloor \frac{d_i}{k} \bigr\rfloor$, $\bigl\lceil \frac{d}{k}
    \bigr\rceil$ hasonló
  + $\bigl\lfloor \frac{d}{k} \bigr\rfloor \le A x \le
    \bigl\lceil \frac{d}{k} \bigr\rceil$, $0 \le x \le (1, 1, \ldots,
    1)^\T$ megoldható, pl.~$x = \bigl( \frac{1}{k}, \frac{1}{k},
    \ldots, \frac{1}{k} \bigr)$ megoldás
    + \aref{ex:linprog:ip:iplp}.~tétel miatt $\exists$ egészértékű
      ($0$-$1$) megoldás \RA ez épp egy jó kiválasztás \qed
+ \dfn a $G$ gráf \emph{perfekt}, ha $\chi(F) = \omega(F)$ $\forall F
  \subseteq G$ feszített részgráfjára
  + $\chi$ a kromatikus szám, $\omega$ a klikkszám
+ \corr minden intervallumgráf perfekt
  + \proof a $G$ intervallumgráf minden feszített részgráfja intervallumgráf
    + elég belátni, hogy $\chi(G) = \omega(G)$
  + alkalmazzuk \aref{thm:liprog:app:interval}.~tételt $k =
    \omega(G)$ választással
    + $\forall i: d_i \le \omega(G)$ \RA $\forall$ klikkben $\forall$ színt
      legfeljebb egyszer használtuk
    + a kapott színezés egy jó színezés \RA $\chi(G) \le k = \omega(G)$
  + mivel minden gráfban $\chi(G) \ge \omega(G)$ \RA $\chi(G) =
    \omega(G)$ \qed

\tetel{A lineáris és egészértékű programozás alkalmazása hálózati
  folyamproblémákra: a maximális folyam, a minimális költségű folyam
  és a többtermékes folyam feladatai, ezek hatékony megoldhatósága a
  tört-, illetve egészértékű esetben.}

+ legyen $G = (V, E)$ irányított gráf, $s, t \in V$, $c\colon E \to
  \RR_{+}$ \emph{kapacitásfüggvény}
  + jelölje $x\colon E \to \RR_{\ge 0}$ $v \in V$-be belépő éleken felvett
    összegét $\rho_x(v)$
  + jelölje $x$ a $v$-ből kilépő éleken felvett összegét $\delta_x(v)$
  + \dfn $x\colon E \to \RR_{\ge 0}$ \emph{folyam}, ha $\forall v \in
    V - \{ s, t\}: \rho_x(v) = \delta_x(v)$
  + \dfn $x$ folyam \emph{megengedett}, ha $\forall e
    \in E : x(e) \le c(e)$
  + \dfn az $x$ megengedett folyam \emph{értéke} $\delta_x(s) - \rho_x(s) =
    \rho_x(t) - \delta_x(t)$
+ \dfn a $C = (S, T)$ \emph{vágás}, ha $S \cup T = V$, $S \cap T =
  \emptyset$, $s \in S$, $t \in T$
  + \dfn a $C = (S, T)$ vágás \emph{értéke} $m_C = \sum_{(x, y) \in E,
    x \in S, y \in T} c(x, y)$
  + tetszőleges $C$ vágás értéke felső becslés minden folyam nagyságára
+ \lemma \label{lem:linprog:folyam:folyam}ha $x\colon E \to \RR_{\ge
  0}$-ra $\forall v \in V - \{s, t\}: \rho_x(v) \ge \delta_x(v)$ és
  $\rho_x(t) \ge \delta_x(s)$ \RA $x$ folyam
  + \proof az $S = \sum_{v \in V - \{s, t\}} \rho_x(v) - \delta_x(v)$
    összeg $\forall$ tagja nemnegatív \RA $0 \le S$
  + vegyük észre, hogy $S = \delta_x(s) - \rho_x(t)$
    + ha $e = (u, v)$, $u \ne s$, $v \ne t$ \RA $x(e)$ pozitív és
      negatív előjellel is megjelenik $S$-ben
    + ha $u = s$, $v \ne t$ \RA $x(e)$ csak $+$ előjellel jelenik meg
      $\rho_x(v)$-nél
    + ha $u \ne s$, $v =e t$ \RA $x(e)$ csak $-$ előjellel jelenik meg
      $\delta_x(u)$-nál
    + az $e = (s, t)$ él nem jelenik meg $S$-ben, és kiesik
      $\delta_x(s) - \rho_x(t)$-ben
  + $0 \le S = \delta_x(s) - \rho_x(t) \le 0$ \RA $S = 0$
    + ez csak úgy lehet, ha $\forall$ (nemnegatív) tag $= 0$ \RA%
      $\forall v \in V - \{ s, t\}: \rho_x(v) = \delta_x(v)$ \qed
+ \prob maximális értékű folyam
  + \DataIn $G = (V, E)$ irányított gráf, $s, t \in V$, $c\colon E \to
    \RR^{+}$ kapacitásfüggvény
  + \DataOut az $x$ maximális értékű folyam
  + legyen $G$ illeszkedési mátrixsza $B$, ennek $v \in V$-hez tartozó
    sora $b_v$
  + legyen $G^* \coloneqq (V, E^*)$, $E^* \coloneqq E \cup \{(t,
    s)\}$, az illeszkedési mátrixsza $B^* = (B|b^*)$
  + ha $x^* = (x|\mu)$ és $B^* x^* \le 0$
    + $\forall v \in V - \{s, t\} : b_v x \le0$ \RA $\delta_x(v) -
      \rho_x(v) \le 0$ \RA $\delta_x(v) \le \rho_x(v)$
    + $s$-re és $t$-re nézve  $\delta_x(s) - \mu \le 0$ és
      $-\rho_x(t) + \mu \le 0$ \RA $\delta_x(s) \le \mu \le \rho_x(t)$
    + \aref{lem:linprog:folyam:folyam}.~lemma szerint $x$ valóban
      folyam, az értéke $\mu$
  + max.~folyam lineáris programja: $\max \{ (0, 0, \ldots, 0, 1) x^*
    : B^* x^* \le 0, x^* \ge 0, x \le c\}$
    + ekvivalensen $\max \{ \mu : (B^* | \text{``$I$''}) (x, \mu | x)
      \le (0 , 0 | c), x \ge 0, \mu \ge 0 \}$
    + a második ``egységmátrixból'' hiányzik az $e^* = (t, s)$-hez
      tartozó oszlop
  + a duális feladatot a kényelmesen a dualitástétel ekvivalens
    alakjából kapjuk
    + $\forall v \in V$ csúcshoz $\pi(v)$ és $\forall e \in E$ élhez
      $w(e)$ változó
    + $\min \{ \sum_{e} w(e) c(e) : \forall v: \pi(v) \ge 0; \forall
      e = (u, v): w(e) \ge 0, \pi(u) - \pi(v) + w(e) \ge 0; \\
      \phantom{\min\{} \pi(t) - \pi(s) \ge 1\}$
+ \thm \label{thm:linprog:folyam:dlp}a fenti duális minimuma épp a maximális hálózati folyam értéke
  + \proof (\LA) tetszőleges $C = (S, T)$ vágáshoz $\exists \pi,
    w: \sum_{e} w(e) c(e) = m_C$
    + legyen $\pi(v) = \text{\footnotesize$\begin{cases}
        0, &\text{ha $v \in S$,} \\
        1, &\text{ha $v \in T$}
      \end{cases}$}$, $w(u, v) = \text{\footnotesize$\begin{cases}
        0, &\text{ha $u \in S$, $v \in T$} \\
        1, &\text{egyébként}
      \end{cases}$}$
    + ez kielégíti a feltételeket, és $\sum_{e} w(e) c(e)$ tényleg a
      vágás értéke
  + (\RA) a duális feladdat mátrixsza TU ($G$ illeszkedési mátrixsza +
    egységvektorok)
    + a jobb oldal egészértékű \RA $w$ és $\pi$ is lehet
      egészértékű az optimális megoldásban
    + készítünk egy $0$-$1$ megoldást:
      $\pi'(v) \coloneqq \text{\footnotesize$\begin{cases}
        0, &\text{ha $\pi(v) \le \pi(s)$,} \\
        1, &\text{ha $\pi(v) > \pi(s)$}
      \end{cases}$}$, $w'(e) \coloneqq \text{\footnotesize$\begin{cases}
        0, &\text{ha $w(e) = 0$,} \\
        1, &\text{ha $w(e) \ge 1$}
      \end{cases}$}$
      + $\pi'(t) - \pi'(s) \ge 0$, mert $\pi(t) - \pi(s) \ge 0$ miatt
        $\pi(t) > \pi(s)$ teljesült
      + ha $\pi'(u) - \pi'(v) + w(u, v) < 0$ \RA $\pi'(u) = 0$, $\pi'(v)
        = 1$, $w(u, v) = 0$
      + de $\pi(u) \le \pi(s) < \pi(v)$, $w(e) = 0$ \RA%
        $\pi(u) - \pi(v) + w(e) < 0$, ellentmondás
    + $\forall e: w'(e) \le w(e)$ \RA $\sum_e w'(e) c(e) \le \sum_e
      w(e) c(e)$ \RA $(\pi', w')$ tényleg optimális
    + $S = \{v : \pi'(v) = 0\}$, $T = \{v : \pi'(v) = 1\}$ egy vágás,
      $m_C = \sum_e w'(e) c(e)$
    + a minimális kapacitású vágás kapacitása tényleg a duál optimuma
      + a dualitástétel miatt ez a primál optimuma
      + ami épp a maximális folyam nagysága \qed
  + beláttuk, hogy a maximális folyam polinomiális időben
    meghatározható
    + pl.~ellipszoid módszerrel
      + de a gyakorlatban Edmonds`--Karp algoritmusa nem használ LP-t
    + $(B^* | \text{``$I$''})$ TU \RA ha $c$ egész, a maximális
      egészértékű folyam is meghatározható
+ \thm (Ford`--Fulkerson) a maximális hálózati folyam nagysága
  megegyezik a minimális kapacitású vágás kapacitásával
  + \proof \aref{thm:linprog:folyam:dlp}.~tétel alapján triviális \qed
+ \prob minimális költségű folyam
  + \DataIn $G = (V, E)$, $s, t \in V$, $c\colon E \to \RR^+$,
    $k\colon E \to \RR_{\ge 0}$ \emph{költségfüggvény}, $M$
    folyamérték
  + \DataOut $x$ legalább $M$ értékű, minimális $\sum_{e \in E} k(e)
    x(e)$ költségű folyam
  + $\max \{ -kx : B^* x^* \le 0, x^* \ge 0, x \le c, \mu \ge M \}$ LP
    feladat \RA polinomiális időben megoldható
  + ha $c$ és $M$ egész \RA az IP verzió is polinomiális, mert a
    mátrix megint TU
+ \prob többtermékes folyamprobléma
  + \DataIn $G = (V, E)$, $(s_1, t_1), (s_2, t_2), \ldots, (s_k,
    t_k)$, $c\colon E \to \RR^+$
  + \DataOut $x_1, x_2, \ldots, x_2$ folyamok, melyekre $\forall e \in
    E: \sum_{i = 1}^k x_i(e) \le c(e)$ és az összes folyamnagyság
    $\sum_{i = 1}^k \delta_{x_i}(s) - \rho_{x_i}(t)$ maximális
  + legyen $B_i^*$ a $G_i^* = (V, E \cup (t_i, s_i))$ illeszkedési
    mátrixsza, $x_i^* = (x_i, \mu_i)$
  + $\max \bigl\{ \sum_{i = 1}^k \mu_i : \forall_{i = 1}^k B_i^* x_i^*
    \le 0, x_i^* \ge 0; \bigl( \sum_{i = 1}^k x_i \bigr) \le c
    \bigr\}$ \RA polinomiális időben megoldható
  + az egészértékű változat $k \ge 2$ esetén nem TU
    + ha $k = 1$ \RA maximális folyam probléma, egész $c$ esetén
      polinomiális
    + ha $k = 2$ \RA a feladat NP-nehéz (\noproof)
