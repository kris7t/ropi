\documentclass[a4paper,11pt,oneside]{article}
\renewcommand*{\baselinestretch}{1.04}
\usepackage[margin=2.5cm]{geometry}

\usepackage[T1]{fontenc}
\usepackage[utf8]{inputenc}
\usepackage{lmodern}

\usepackage{paralist}
\pltopsep=0pt
\plitemsep=0pt
\plparsep=0pt

\def\+{+}

\makeatletter
\catcode`\ =12\let\@nl@space= \catcode`\ =10
\newcount\@nl@rlevel
\newcount\@nl@llevel
\@nl@llevel=-1

\def\@nl{%
  \catcode`\ =12
  \global\@nl@rlevel=0
  \futurelet\@nl@store\@nl@%
}
\def\@nl@gobble#1{\futurelet\@nl@store\@nl@}
\def\@nl@enditemize{
  \ifnum\the\@nl@rlevel<\the\@nl@llevel%
    \end{compactitem}%
    \egroup%
    \expandafter\@nl@enditemize%
  \else%
    \ifnum\the\@nl@rlevel=\the\@nl@llevel\else%
       \errmessage{Error: inconsistent identation}
    \fi%
  \fi%
}
\def\@nl@{%
  \ifx\@nl@store\@nl@space%
    \global\advance\@nl@rlevel by 1
    \expandafter\@nl@gobble%
  \else%
    \catcode`\ =10
    \ifx\@nl@store+%
      \ifnum\the\@nl@rlevel>\the\@nl@llevel%
        \bgroup%
        \@nl@llevel=\the\@nl@rlevel
        \begin{compactitem}%
      \fi%
      \@nl@enditemize%
      \item \expandafter\expandafter\expandafter\@gobble%
    \else%
      \ifx\@nl@store\@nl%
        \global\@nl@rlevel=-1\relax\@nl@enditemize\par
      \else\space\fi%
    \fi%
  \fi%
}
\makeatother

\def\magyarOptions{defaults=prettiest}
\usepackage[magyar]{babel}

\usepackage{amsmath,wasysym,mathtools,mleftright,relsize,amssymb}

\usepackage[stretch=10]{microtype}

\usepackage{xspace,mdframed,array}

\newmdenv{tetelframe}
\setcounter{secnumdepth}{-1}
\newcounter{tetel}
\newcommand{\tetel}[1]{\begin{tetelframe}\noindent\refstepcounter{tetel}\textbf{\thetetel.}~#1\end{tetelframe}}

\newcommand*{\ra}{\ensuremath{\rightarrow}\xspace}
\newcommand*{\RA}{\ensuremath{\Longrightarrow}\xspace}
\newcommand*{\LA}{\ensuremath{\Longleftarrow}\xspace}
\newcommand*{\app}{\ensuremath{\approx}\xspace}
\newcommand*{\lra}{\ensuremath{\leftrightarrow}\xspace}
\newcommand*{\LRA}{\ensuremath{\Longleftrightarrow}\xspace}

\newcounter{theorem}[tetel]
\renewcommand*{\thetheorem}{\thetetel.\arabic{theorem}}
\newcommand*{\theoremlike}[1]{\refstepcounter{theorem}{\textbf{\thetheorem.~#1:}}\xspace}
\newcommand*{\thm}{\theoremlike{tétel}}
\newcommand*{\example}{\theoremlike{példa}}
\newcommand*{\prob}{\theoremlike{probléma}}
\newcommand*{\alg}{\theoremlike{algoritmus}}
\newcommand*{\dfn}{\theoremlike{definíció}}
\newcommand*{\lemma}{\theoremlike{lemma}}
\newcommand*{\corr}{\theoremlike{következmény}}
\newcommand*{\proof}{\textit{bizonyítás:}\xspace}
\newcommand*{\noproof}{\textit{nem bizonyítjuk}}
\newcommand*{\qed}{\hfill\ensuremath{\square}}
\newcommand*{\RR}{\mathbb{R}}
\newcommand*{\ZZ}{\mathbb{Z}}
\newcommand*{\Nat}{\mathbb{N}}
\newcommand*{\GF}{\textrm{GF}_2}
\newcommand*{\MM}{\mathcal{M}}
\newcommand*{\NN}{\mathcal{N}}
\newcommand*{\UU}{\mathcal{U}}
\newcommand*{\FF}{\mathcal{F}}
\newcommand*{\BB}{\mathcal{B}}
\newcommand*{\CC}{\mathcal{C}}
\newcommand*{\T}{\mathrm{T}}
\newcommand*{\IP}{\textit{IP}}
\newcommand*{\LP}{\textit{LP}}
\newcommand*{\DLP}{\textit{DLP}}
\newcommand*{\DIP}{\textit{DIP}}

\newcommand*{\DataIn}{\textsc{Input}:\xspace}
\newcommand*{\DataOut}{\textsc{Output}:\xspace}

\newcommand{\vertle}{\rotatebox{90}{$\mkern-2mu\le$}}
\newcommand{\vertgt}{\rotatebox{90}{$\mkern-2mu>$}}
\DeclareMathOperator*{\argmax}{arg\,max}

\usepackage{tikz}
\usetikzlibrary{positioning,calc,shapes.geometric,through}
\let\tikzpictureunpatched\tikzpicture
\def\tikzpicture{\catcode`\^^M=5\tikzpictureunpatched}
\tikzset{
  vertex/.style={shape=circle,fill=black,inner sep=0pt,
    font=\nullfont,minimum width=4pt}
}

\title{Rendszeroptimalizálás vizsgatételek (2015/2016.~második
  félév)}
\author{Marussy Kristóf}

%\makeatletter
%\usepackage[pdftitle={\@title},pdfauthor={\@author}]{hyperref}
%\makeatother

\begin{document}

\makeatletter
\catcode`\^^M=\active%
\let^^M=\@nl%
\makeatother

\maketitle

\section{Lineáris programozás}

\tetel{Az optimális hozzárendelés problémája, Egerváry algoritmusa.}

\tetel{A lineáris programozás alapfeladata, kétváltozós feladat
  grafikus megoldása. Lineáris egyenlőtlenségrendszer megoldása
  Fourier`--Motzkin eliminációval.}

\tetel{Farkas-lemma (két alakban). A lineáris program célfüggvénye
  felülről korlátosságának feltételei.}

\tetel{A lineáris programozás dualitástétele (két alakban). A lineáris
  programozás alapfeladatának bonyolultsága (biz.~nélkül).}

\tetel{Egészértékű programozás: a feladat bonyolultsága, korlátozás és
  szétválasztás (Branch and Bound). Totálisan unimoduláris mátrix
  fogalma, példák. Egészértékű programozás totálisan unimoduláris
  együtthatómátrixszal (biz.~nélkül).}

\tetel{A lineáris és egészértékű programozás alkalmazása páros
  gráfokra és intervallumrendszerekre: Egerváry tétele,
  intervallumrendszerek egyenletes színezése.}

\tetel{A lineáris és egészértékű programozás alkalmazása hálózati
  folyamproblémákra: a maximális folyam, a minimális költségű folyam
  és a többtermékes folyam feladatai, ezek hatékony megoldhatósága a
  tört-, illetve egészértékű esetben.}

\section{Matroidok}

\tetel{Matroid definíciója, alapfogalmak (bázis, rang, kör). Példák:
  lineáris matroid (mátrixmatroid), grafikus matroid, uniform
  matroid. A rangfüggvény szubmodularitása.}

\tetel{Mohó algoritmus matroidon. Matroid megadása rangfüggvényével,
  bázisaival (biz.~nélkül).  Matroid duálisa, a duális matroid
  rangfüggvénye.}

\tetel{Elhagyás és összehúzás. Matroidok direkt összege,
  összefüggősége. $T$ test felett reprezentálható matroid duálisának $T$
  feletti reprezentálhatósága.}

\tetel{Grafikus, kografikus, reguláris, bináris és lineáris matroid
  fogalma, ezek kapcsolata (ebből bizonyítás csak a grafikus és
  reguláris matroidok közötti kapcsolatra), példák. Fano-matroid,
  példa nemlineáris matroidra. Bináris, reguláris és grafikus
  matroidok jellemzése tiltott minorokkal: Tutte tételei
  (biz.~nélkül).}

\tetel{Matroidok összege. $k$-matroid-metszet probléma, ennek
  bonyolultsága $k \ge 3$ esetén.}

\tetel{A $k$-matroid partíciós probléma, ennek algoritmikus
  megoldása. A $2$-matroid-metszet feladat visszavezetése matroid
  partíciós problémára.}

\tetel{$k$-polimatroid rangfüggvény fogalma. A
  $2$-polimatroid-matching probléma, ennek bonyolultsága, Lovász
  tétele (biz.~nélkül).}

\section{Közelítő és ütemezési algoritmusok}

\tetel{Polinomiális időben megoldható feladat fogalma, példák. Az NP,
  co-NP, NP-nehéz és NP-teljes problémaosztályok definíciója,
  viszonyaik, példák problémákra valamennyi osztályból. NP-nehéz
  feladatok polinomiális speciális esetei: algoritmus a maximális
  független ponthalmaz problémára és az élszínezési problémára páros
  gráfokon. Additív hibával közelítő algoritmusok speciális pont-,
  illetve élszínezési problémákra.}

\tetel{A Hamilton-kör probléma visszavezetése a leghosszabb kör
  probléma additív közelítésére. $k$-approximációs algoritmus fogalma,
  példák: két-két algoritmus a minimális lefogó ponthalmaz keresésére
  és a maximális páros részgráf keresésére. Minimális levelű, illetve
  maximális belső csúcsú feszítőfa keresése. Approximációs algoritmus
  az utóbbi feladatra (biz.~nélkül).}

\tetel{A minimális lefogó ponthalmaz probléma visszavezetése a
  halmazfedési feladatra, a halmazfedési feladat közelítése, éles
  példa. Közelítő algoritmus a Steiner-fa problémára, éles példa.}

\tetel{A Hamilton-kör probléma visszavezetése az általános utazóügynök
  probléma $k$-app\-ro\-xi\-má\-ci\-ós megoldására. Közelítő
  algoritmusok a metrikus utazóügynök problémára, Christofides
  algoritmusa.}

\tetel{Teljesen polinomiális approximációs séma fogalma. A részösszeg
  probléma, bonyolultsága. Teljesen polinomiális approximációs séma a
  részösszeg problémára.}

\tetel{Ütemezési feladatok típusai. Az
  $1 | \textrm{prec} | C_{\textrm{max}}$ és az $1 \| \Sigma C_j$
  feladat. Approximációs algoritmusok a $P \| C_{\textrm{max}}$
  feladatra: listás ütemezés tetszőleges sorrendben, éles példa
  tetszőleges számú gép esetére; listás ütemezés LPT sorrendben
  (biz.~nélkül), éles példa tetszőleges számú gép
  esetére. Approximációs algoritmus a
  $P | \textrm{prec} |C_{\textrm{max}}$ feladatra (biz.~nélkül),
  példák: az LPT sorrend, illetve a leghosszabb út szerinti ütemezés
  sem jobb, mint $(2 - \frac{m}{1})$-approximáció. A
  $P | \textrm{prec}, p_i = 1 | C_{\textrm{max}}$ feladat, Hu
  algoritmusa (biz.~nélkül).}

\section{Megbízható hálózatok tervezése}

\tetel{Globális és lokális élösszefüggőség és élösszefüggőségi szám
  fogalma, Menger irányítatlan gráfokra és élösszefüggőségre vonatkozó
  két tétele (biz.~nélkül). $\lambda(G)$ meghatározása folyamok
  segítségével négyzetes és lineáris számú folyamkereséssel.}

\tetel{$\lambda(G)$ meghatározása összehúzások segítségével, Mader
  tétele, Nagamochi és Ibaraki algorit- musa.}

\tetel{Minimális méretű $2$-élösszefüggő részgráfok keresése. A probléma
  NP-nehézsége, Khuller`--Vishkin algoritmus (biz.~nélkül).}

\section{Hálózatelméleti alkalmazások}

\tetel{Kirchhoff tételei a klasszikus villamos hálózatok analízisére.}

\tetel{Kirchhoff eredményeinek általánosítása transzformátorokat vagy
  girátorokat is tartalmazó hálózatokra (biz.~nélkül). Algoritmusok a
  feltételek ellenőrzésére.}

\tetel{Kirchhoff eredményeinek általánosítása: szükséges feltétel
  tetszőleges lineáris sok-ka\-pu\-kat is tartalmazó hálózatok egyértelmű
  megoldhatóságára. Villamos hálózatok duálisa.}

\section{Statikai alkalmazások}

\tetel{Rúdszerkezetek, merevségi mátrix, merevség, egyszerű rácsos
  tartók, Cremona`--Maxwell diagramok.}

\tetel{Minimális merev rúdszerkezetek általános helyzetben, Laman
  tétele (biz.~nélkül), Lovász és Yemini tétele.}

\tetel{Síkbeli négyzetrácsok és egyszintes épületek átlós merevítése.}

\end{document}