\documentclass[a4paper,11pt,oneside]{article}
\renewcommand*{\baselinestretch}{1.04}
\usepackage[margin=2.5cm]{geometry}

\usepackage[T1]{fontenc}
\usepackage[utf8]{inputenc}
\usepackage{lmodern}

\usepackage{paralist}
\pltopsep=0pt
\plitemsep=0pt
\plparsep=0pt

\def\+{+}

\makeatletter
\catcode`\ =12\let\@nl@space= \catcode`\ =10
\newcount\@nl@rlevel
\newcount\@nl@llevel
\@nl@llevel=-1

\def\@nl{%
  \catcode`\ =12
  \global\@nl@rlevel=0
  \futurelet\@nl@store\@nl@%
}
\def\@nl@gobble#1{\futurelet\@nl@store\@nl@}
\def\@nl@enditemize{
  \ifnum\the\@nl@rlevel<\the\@nl@llevel%
    \end{compactitem}%
    \egroup%
    \expandafter\@nl@enditemize%
  \else%
    \ifnum\the\@nl@rlevel=\the\@nl@llevel\else%
       \errmessage{Error: inconsistent identation}
    \fi%
  \fi%
}
\def\@nl@{%
  \ifx\@nl@store\@nl@space%
    \global\advance\@nl@rlevel by 1
    \expandafter\@nl@gobble%
  \else%
    \catcode`\ =10
    \ifx\@nl@store+%
      \ifnum\the\@nl@rlevel>\the\@nl@llevel%
        \bgroup%
        \@nl@llevel=\the\@nl@rlevel
        \begin{compactitem}%
      \fi%
      \@nl@enditemize%
      \item \expandafter\expandafter\expandafter\@gobble%
    \else%
      \ifx\@nl@store\@nl%
        \global\@nl@rlevel=-1\relax\@nl@enditemize\par
      \else\space\fi%
    \fi%
  \fi%
}
\makeatother

\def\magyarOptions{defaults=prettiest}
\usepackage[magyar]{babel}

\usepackage{amsmath,wasysym,mathtools,mleftright,relsize,amssymb}

\usepackage[stretch=10]{microtype}

\usepackage{xspace,mdframed,array}

\newmdenv{tetelframe}
\setcounter{secnumdepth}{-1}
\newcounter{tetel}
\newcommand{\tetel}[1]{\begin{tetelframe}\noindent\refstepcounter{tetel}\textbf{\thetetel.}~#1\end{tetelframe}}

\newcommand*{\ra}{\ensuremath{\rightarrow}\xspace}
\newcommand*{\RA}{\ensuremath{\Longrightarrow}\xspace}
\newcommand*{\app}{\ensuremath{\approx}\xspace}
\newcommand*{\lra}{\ensuremath{\leftrightarrow}\xspace}

\newcounter{theorem}[tetel]
 \renewcommand*{\thetheorem}{\thetetel.\arabic{theorem}}
\newcommand*{\theoremlike}[1]{\refstepcounter{theorem}{\textbf{\thetheorem.~#1:}}\xspace}
\newcommand*{\thm}{\theoremlike{tétel}}
\newcommand*{\example}{\theoremlike{példa}}
\newcommand*{\prob}{\theoremlike{probléma}}
\newcommand*{\alg}{\theoremlike{algoritmus}}
\newcommand*{\dfn}{\theoremlike{definíció}}
\newcommand*{\lemma}{\theoremlike{lemma}}
\newcommand*{\corr}{\theoremlike{következmény}}
\newcommand*{\proof}{\textit{bizonyítás:}\xspace}
\newcommand*{\qed}{\hfill\ensuremath{\square}}
\newcommand*{\RR}{\mathbb{R}}
\newcommand*{\T}{\mathrm{T}}

\newcommand*{\DataIn}{\textsc{Input}:\xspace}
\newcommand*{\DataOut}{\textsc{Output}:\xspace}

\title{Rendszeroptimalizálás vizsgatételek (2015/2016.~második
  félév)}
\author{Marussy Kristóf}

%\makeatletter
%\usepackage[pdftitle={\@title},pdfauthor={\@author}]{hyperref}
%\makeatother

\begin{document}

\makeatletter
\catcode`\^^M=\active%
\let^^M=\@nl%
\makeatother

\maketitle
\thispagestyle{empty}

\begin{ocg}[printocg=never]{ForkMe}{1}{1}
  \begin{tikzpicture}[remember picture, overlay]
    \node[rotate=-45, shift={(0, -3.2cm)}] at (current page.north east) {%
      \begin{tikzpicture}[remember picture, overlay]
        \node at (0pt, 0pt) {\pgfuseshading{forkmeshading}};
        \node[text=forkmefg] at (0pt, 0pt) {%
          \large\bfseries\clear Fork me on GitHub};
        \draw[forkmefg!70!forkmebg, dashed, line width=.08em, dash pattern=on .5em off 1.5\pgflinewidth] (-600pt,12pt) rectangle (600pt,-12pt);
      \end{tikzpicture}%
    };
    \node [anchor=north east] at (current page.north east) {
      \href{https://github.com/kris7t/ropi}{\parbox[t][5cm]{5cm}{\hfill}}};
  \end{tikzpicture}
\end{ocg}


\section{Lineáris programozás}

\tetel{Az optimális hozzárendelés problémája, Egerváry algoritmusa.}

+ \example egy cég számos megrendelést kap különboző ``egyszemélyes'',
  egyforma idő alatt elvégezhető munkák elvégzésére
  + kimutatást készítünk arról, hogy melyik dolgozó melyik munkát
    tudja elvégezni
  + cél a profit maximalizálása (lehető legtöbb munka elvégzése)
+ \alg ``magyar módszer''
  + \DataIn $G = (F, L; E)$ páros gráf
  + \DataOut $M \subseteq G$ egy maximális méretű párosítás
  + induljunk ki egy tetszőleges (pl.~az üres) $M$ párosításból
  + \emph{alternáló út} = párosítatlan $F$-beli csúcsból indul, $\forall$
    második éle az $M$-hez tartozik
    + \emph{javító út} = olyan alternáló út, ami párosítatlan $L$-beli
      pontban ér véget
  + amíg találunk $J$ javítóutat (pl.~szélességi kereséssel) \RA $M
    \gets M - (J \cap M) \cup (J - M)$
+ \thm a ``magyar módszer'' valóban maximális párosítást talál
  $G$-ben
  + \proof $M \coloneqq \text{a párosítás, amit az algoritmus
    megtalált}$
  \begin{align*}
    F_1 &\coloneqq F - M, \text{az $M$ által le nem fedett $F$-beli
          pontok halmaza,}\\
    L_2 &\coloneqq \text{az $F_1$-ből alternáló úton elérhető pontok
          halmaza,} \\
    F_2 &\coloneqq \text{az $L_2$-beli pontok $M$ szerinti párjai,} \\
    L_3 &\coloneqq \text{az $F_1$-ből alternáló úton nem elérhető
          $L$-beli pontok halmaza,} \\
    F_3 &\coloneqq \text{az $L_3$-beli pontok $M$ szerinti párjai,} \\
    L_1 &\coloneqq L - M, \text{az $M$ által le nem fedett $L$-beli pontok
          halmaza}
  \end{align*}
  + vegyük észre, hogy $G$-ben nem vezethet $F_1 \cup F_2$ és $L_1
    \cup L_3$ között él
    \par
    {\centering\begin{tabular}{c|p{6cm}|p{6cm}}
      & \multicolumn{1}{c|}{$F_1$} & \multicolumn{1}{c}{$F_2$} \\\hline
      $L_1$ & $1$ hosszú javító út lenne & $\ge 3$ hosszú javító út lenne \\\hline
      $L_3$ & az $L_3$-beli csúcs $1$ hosszú alternáló úton elérhető lenne, azaz $L_2$-beli
            & az $L_3$-beli csúcs $\ge 3$ hosszú alternáló úton elérhető lenne, azaz $L_2$-beli
    \end{tabular}\par}
  + $F_1 \cup F_2$ $\forall$ szomszédja $L_2$-beli, $L_2 \cup F_3$ egy lefogó ponthalmaz
  + mivel épp $\lvert M \rvert = \lvert L_2 \cup F_3 \rvert$, $M$
    valóban maximális méretű \qed
+ \prob optimális hozzárendelés
  + \DataIn $G = (F, L; E)$ páros gráf, $w\colon E \to \RR$
    súlyfüggvény
  + \DataOut $M \subseteq G$ párosítás úgy, hogy $\sum_{e \in M} w(e)$
    maximális
  + az optimális hozzárendelés megoldható maximális súlyú
    \emph{teljes} párosítás keresésével
    + ha $\lvert F \rvert \ne \lvert L \rvert$, adjunk $G$-hez annyi
      csúcsot, hogy egyenlőek legyenek
    + legyen $G' = (F, L; E')$ teljes páros gráf ($E' = F
      \times L$), $w'(e) \coloneqq \text{\footnotesize$\begin{cases}
        w(e), &\text{ha $e \in E$,} \\
        0, & \text{ha $e \notin E$}
      \end{cases}$}$
    + $G'$ egy $M'$ maximális súlyú teljes párosítasa a $G$ maximális
      súlyú párosítása az $E' - E$ élek elhagyása után
    + $G$ maximális súlyú $M$ párosításához megfelelő $E' - E$ éleket
      hozzávéve $G'$ maximális súlyú teljes párosítását kapjuk
+ \dfn a $c\colon F \cup L \to \RR$ függvény \emph{címkézés} a $G =
  (F, L; E)$ páros gráfra és a $w\colon E \to \RR$ súlyfüggvényre
  nézve, ha $\forall e = \{ x, y \} \in E\colon c(x) + c(y) \ge w(e)$
+ \lemma a $w$ súlyfüggvénnyel súlyozott $G = (F, L; E)$ páros gráf
  tetszőleges $M$ teljes párosítására és tetszőleges $c$
  címkézésére igaz, hogy $\sum_{e \in M} w(e) \le \sum_{v \in F
    \cup L} c(v)$
  + \proof $\displaystyle \sum_{e \in M} w(e) \le \sum_{\mathclap{e = \{f, l\} \in M}}\, c(f) +
    c(l) \overset{\substack{\text{$\forall$ csúcs legfeljebb}\\\text{$1$-szer szerepel $M$-ben}}}{\le} \sum_{f \in F} c(f) + \sum_{l \in L} c(l) = \sum_{\mathclap{v \in F
    \cup L}}\, c(v)$ \qed
  + \dfn a $G$ gráf $e = \{x, y\}$ éle \emph{piros}, ha $c(x) +
    c(y) = w(e)$
  + \corr ha az $M$ teljes párosítás $\forall$ éle piros valamely $c$
    címkézésre nézve, akkor $M$ maximális súlyú teljes párosítás
+ \alg Egerváry algoritmusa
  + \DataIn $G = (F, L; E)$ páros gráf, $w\colon E \to \RR$
  súlyfüggvény
  + \DataOut $M \subseteq G$ \emph{teljes} párosítás úgy, hogy $\sum_{e \in M} w(e)$ maximális
  + 0.~lépés: legyen $M = \emptyset$, $c(v) = \begin{cases}
    \max_{y \in L, \{v, y\} \in E} w({v, y}), & \text{ha $v \in F$,} \\
    0, & \text{ha $v \in L$}
  \end{cases}$
  + 1.~lépés: a javító utas algoritmussal keressünk bővítsük $M$-et
    maximális élszámú párosítássá a piros részgráfban
    + ha $M$ most már teljes \RA STOP, $M$ a keresett párosítás, $c$ a
    keresett címkézés, $w(M) = c(M)$
  + 2.~lépés: $\delta \coloneqq \min \{ c(x) + c(y) - w(\{x, y\}) \mid
    \{x, y\} \in E, x \in F_1 \cup F_2, y \in L_1 \cup L_3 \}$
    + állítsuk elő a $c(v) \gets \text{\footnotesize$\begin{cases}
      c(v) - \delta, & \text{ha $v \in F_1 \cup F_2$,} \\
      c(v) + \delta, & \text{ha $v \in L_2$,} \\
      c(v), & \text{ha $v \in F_3 \cup L_1 \cup L_3$}
    \end{cases}$}$ új címkézést, majd GOTO 1.
+ \thm Egerváry algortimusa $O(n^2 e)$ lépésben maximális súlyú teljes
  párosítást állít elő
  + \proof a 0.~lépésben megadott $c$ valóban címkézés
  + a 2.~lépésben $\delta$ kiszámításához valóban van él $F_1 \cup
  F_2$ és $L_1 \cup L_3$ között
    + ha nem lenne, $N(F_1 \cup F_2) = L_2$
    + $\lvert F_1 \cup F_2 \rvert > \lvert F_2 \rvert = \lvert L_2
    \rvert$ miatt ekkor nem teljesül a Hall-feltétel \RA $\nexists$
    teljes párosítás!
  + a 2.~lépés után is címkézés marad $c$
    + csak az $F_1 \cup F_2$ és $L_1 \cup L_3$ között vezető élekre
    csökken $c(x) + c(y)$
    \par
    {\centering
      \begin{tabular}{c|c|c|c}
        & $F_1$ & $F_2$ & $F_3$ \\\hline
        $L_1$ & $-\delta$ (nem lehet piros) & $-\delta$ (nem lehet piros) & $0$ \\\hline
        $L_2$ & $-\delta + \delta$ & $-\delta + \delta$ & $+\delta$
                                                          (piros eltűnhet) \\\hline
        $L_3$ & $-\delta$ (nem lehet piros) & $-\delta$ (nem lehet piros) & $0$
      \end{tabular}\par}
    + $\delta$ definíciója garantálja, hogy továbbra is $c(x) + c(y)
      \ge w(\{x, y\})$
  + $M$ élei a 2.~lépés után is pirosak
    + csak $x \in F_3$, $y \in L_2$ élek színeződhetnek vissza (itt nőtt
      $c(x) + c(y)$)
    + ezek nem lehetnek $M$ élei, mert $F_3$ párja $L_3$, $L_2$ párja
    $F_2$
    + továbbra is van $F_1$-ből piros alternáló út $L_2$ csúcsaiba
  + egy iterációban vagy $M$, vagy $L_2$ elemszáma nő
   + $O(n)$ lépés után $M$ elemszáma mindenképp nő, mert ekkorra már $L_2$
     lefedné $L$-t
   + $O(n^2)$ iterációban $M$ maximális párosítás lesz
   + ezért $O(n^2 e)$ időben az algoritmus véget ér \qed

\tetel{A lineáris programozás alapfeladata, kétváltozós feladat
  grafikus megoldása. Lineáris egyenlőtlenségrendszer megoldása
  Fourier`--Motzkin eliminációval.}

+ legyen $A \in \RR^{m \times n}$, $x \in \RR^n$, $b \in \RR^m$, $c \in
  \RR^n$ \RA \emph{lineáris program}
+ \dfn \emph{linerási programozás alapfeladata}: $\min_{x} \{ cx : Ax
  \le b \}$
+ kétdimenziós feladat megoldása
  + az $p_1 x_1 + p_2 x_2 \ge k$ alakú feltétel $p_1 x_1 + p_2
  x_2 = k$ egyenese két félsíkra bontja a síkot
    + a $\ge$ jelnek megfelelő félsíkok metszete adja megengedett
      megoldások tartományát
  + ha $c = (q_1, q_2)$ \RA $q_1 x_1 + q_2 x_2 = 0$-val párhuzamos
  egyeneseket húzunk
    + minden egyenesre kiszámítjuk a célfüggvény értékét
    + a maximumhely a legnagyobb egyenes és a félsíkok metszetének
    közös pontja
  + általánosítás: \emph{hipersíkok} által határolt \emph{poliéder}
+ ekvivalens átalakítások
  + egyenlőtlenség megszorzása \emph{pozitív} számmal
  + két egyenlőtlenség összegének hozzávétele az egyenlőtlenségrendszerhez
  + \emph{nem} ekvivalens: szorzás negatív számmal
+ \alg Fourier`--Motzkin elimináció
  + $n$ változós lineáris program visszavezetése egy $n - 1$ változós
    $A^*$, $b^*$ lineáris programra
  + végül $1$ változós lineáris program megoldása
  + $(A|b)$ bővített együttható mátrix
    + szorozzuk pozitív számokkal a
      sorokat, hogy az $1$.~oszlopban csak $-1, 0, 1$ legyen
  + legyen $I$, $J$, $K$ rendre az $1$-gyel, $-1$-gyel és $0$-val
    kezdődő sorok indexeinek halmaza
  + $\RR^{\lvert K \rvert \times n} \ni (A_0|b_0) =$ $(A|b)$ $0$-val
    kezdődő sorai az első oszlop elhagyásával
  + $Ax \le b$ egy megoldása $x = (\lambda, \bar{x})$ alakú \RA $A_0
    \bar{x} \le b_0$ \RA mikor van megfelelő $\lambda$?
  + ha $J = \emptyset$ (nincs $-1$-gyel kezdődő sor)
    + $\forall i \in I : \lambda + \bar{a}_i \bar{x} \le b_j$ \RA
      $\lambda \le \min_{i \in I} b_i - \bar{a}_i \bar{x}$, ami mindig
      kielégíthető
    + $A^*, b^* =$ az $\bar{A}$-ból és $b$-ből az $i \in I$ indexű sorok
      elhagyásával kapott rendszer
  + ha $I = \emptyset$ (nincs $1$-gyel kezdődő sor)
    + $\forall j \in J : -\lambda + \bar{a}_j \bar{x} \le b_j$ \RA
      $\lambda \ge \max_{j \in J} \bar{a}_j \bar{x} - b_j$, ami mindig
      kielégíthető
    + $A^*, b^* =$ az $\bar{A}$-ból és $b$-ből az $j \in J$ indexű sorok
      elhagyásával kapott rendszer
  + ha $I \ne \emptyset$ és $J \ne \emptyset$ \RA
    $\forall i \in I : \lambda + \bar{a}_i \bar{x} \le b_j$ és
    $\forall j \in J : -\lambda + \bar{a}_j \bar{x} \le b_j$
    + $\forall i \in I, j \in J: \bar{a}_j \bar{x} - b_j \le \lambda
      \le \bar{b}_i - \bar{a}_i \bar{x}$ \RA $\forall i \in I, j \in J:
      (\bar{a}_i + \bar{a}_j) \bar{x} \le b_i + b_j$
    + $A^*, b^* =$ az $\bar{A}$ és $b$-ből $i \in I$ és $j \in J$
      indexű sorait az összes lehetséges módon összeadjuk, a $k \in K$
      indexű sorokat hozzávesszük és a csupa $0$ első oszlopot
      elhagyjuk
  + egyváltozós rendszer megoldása
    + ha $\exists k \in K : b_k < 0$ \RA a rendszer nem megoldható
    + ha $\max_{j \in J} -b_j > \min_{i \in I} b_i$ \RA a rendszer nem
      megoldható
    + egyébként \RA $x_1 \in \bigl[\max_{j \in J} -b_j, \min_{i \in I}
      b_i\bigr]$ megoldás \qed

\tetel{Farkas-lemma (két alakban). A lineáris program célfüggvénye
  felülről korlátosságának feltételei.}

+ \lemma Farkas-lemma, 1.~alak
  + tetszőleges $A$ és $b$ esetén az alábbi két rendszer közül
    pontosan egynek van megoldása:\\
    (1)~$Ax \le b$\qquad(2)~$yA = 0$, $y \ge 0$, $yb < 0$
+ \proof (1)~megoldható \RA (2)~nem megoldható
  + $0 = 0x = (yA) x = y (Ax) \le yb < 0$, ellentmondás
+ (1)~nem megoldható \RA (2)~megoldható
  + alkalmazzuk a Fourier`--Motzkin eliminációt az
    (1)~egyenletrendszerre
  + a feltevés szerint a kapott egyváltozós $(A^*|b^*)$ rendszer nem
    megoldható
    + ha van $(0|\beta)$, $\beta < 0$ sor
      + $\exists$ $(A|b)$ sorainak olyan nemnegatív $y$ együtthatós
      lin.~kombinációja, hogy\\
        $y (A|b) = (0, 0, \ldots, 0|\beta)$ \RA $y$ kielégíti (2)-t
    + ha van $(-1, \beta_j)$ és $(1,\beta_i)$ sor úgy, hogy $-\beta_j >
    \beta_i$ \RA $\beta = \beta_i + \beta_j < 0$
      + $\exists y_j \ge 0: y_j (A|b) = (0, 0, \ldots, 0, -1 | \beta_j)$
        és $\exists y_i \ge 0: y_i (A|b) = (0, 0, \ldots, 0, 1 | \beta_i)$
      + ekkor $y = y_i + y_j \ge 0$, $y (A|b)
      = (0, 0, \ldots, 0 | \beta)$ \RA $y$ kielégíti (2)-t \qed
+ \lemma Farkas-lemma, 2.~alak
  + tetszőleges $A$ és $b$ esetén az alábbi két rendszer közül
    pontosan egynek van megoldása:\\
    (1)~$Ax = b$, $x \ge 0$\qquad(2)~$yA \ge 0$, $yb < 0$
  + \proof (1)~megoldható \RA (2)~nem megoldható
    + $0 \le (yA)x = y(Ax) = yb < 0$, ellentmondás
  + (2)~nem megoldható \RA (1)~megoldható
    + vizsgáljuk az ekvivalens $y(-A) \le 0$, $yb = -1 < 0$  rendszert
    + tömör alakban $(-A|b|-b)^\T y \le (0, 0, \ldots, 0, -1, 1)^\T$
    + a lemma 1.~alakja szerint $\exists x: x (-A|b|-b)^\T = 0, x
      \ge 0, x (0, 0, \ldots, 0, -1, 1)^\T < 0$
    + legyen $x = (\bar{x} | \lambda, \mu)$ \RA $-A\bar{x} + (\lambda -
      \mu)b = 0$, $\bar{x} \ge 0$, $\lambda, \mu \ge 0$, $-\lambda +
      \mu < 0$
    + ekkor $A\frac{\bar{x}}{\lambda - \mu} = b$ és $\lambda - \mu > 0$
      miatt $\frac{\bar{x}}{\lambda - \mu} \ge 0$ \RA%
      $\frac{\bar{x}}{\lambda - \mu}$ kielégíti (1)-et \qed
+ \thm \label{thm:linprog:farkas:3kalicka}ha $Ax \le b$ megoldható,
  $c$ tetszőleges \RA az alábbi állítások ekvivalensek
  + (1)~az $Ax \le b$ megoldáshalmazán $cx$ felülről korlátos
  + (2)~nincs megoldása az $Az \le 0$, $cz > 0$ rendszernek
  + (3)~van megoldása az $yA = c$, $y \ge 0$ rendszernek
  + \proof (1) \RA (2), legyen $x_0$ (1) egy megoldása és
    indir.~tfh.~$z$ (2) egy megoldása
    + ekkor $A(x_0 + \lambda z) \le b$, de $\lambda > 0$ esetén
      $c(x_0 + \lambda z) = c x_0 + \lambda c z$ tetszőlegesen nagy
    + $cx$ nem felülről korlátos \RA ellentmondás
  + (2) \RA (3), tekintsük a (nem megoldható) $z (-A)^\T \ge 0$, $z
    (-c) < 0$ rendszert
    + a Farkas-lemma 2.~alakja szerint $\exists y: (-A)^\T y = -c, y \ge 0$
    + $(-A)^\T y = -c$ \RA $A^\T y = y A = c$ \RA ez az $y$ épp
      kielégíti a (3)-as rendszert
  + (3) \RA (1), legyen $y$ a (3)-as rendszer egy megoldása
    + $cx = (yA)x = y (Ax) \overset{y \ge 0, Ax \le b}{\le} yb$ \RA%
    $yb$ a $cx$ egy felső korlátja \qed

\tetel{A lineáris programozás dualitástétele (két alakban). A lineáris
  programozás alapfeladatának bonyolultsága (biz.~nélkül).}

+ \thm \label{thm:linprog:dual:dual}a lineáris programozás dualitástétele
  + ha $\max \{ cx : Ax \le b \}$ (primál) program megoldható és
    felülről korlátos, akkor
    + (1)~$\min \{ yb : yA = c, y \ge 0 \}$ (duális) program megoldható és
      alulról korlátos
    + (2)~a primál programnak $\exists$ maximuma és a duális programnak
      $\exists$ minimuma
    + (3)~$\max \{ cx : Ax \le b \} = \min \{ yb : yA = c, y \ge 0 \}$
+ \lemma \label{lem:linprog:dual:t}legyen $A x \le b$ megoldható, $t
  \in \RR$, de $A x \le b$, $cx \ge t$ nem megoldható
  + ekkor a $yA = c$, $y \ge 0$, $yb < t$ rendszer megoldható
  + \proof alkalmazzuk a Farkas-lemmát a $(A|-c) x \le (b|-t)$-re, $y
    \coloneqq (\bar{y}|\lambda)$
    + $\bar{y} A - \lambda c = 0$, $\bar{y} \ge 0$, $\lambda \ge 0$,
      $\bar{y} b - \lambda t < 0$
    + ha $\lambda = 0$ \RA $0 = 0 x = (\bar{y} A) x = \bar{y} (Ax)
      \overset{\bar{y} \ge 0, Ax \le b}{\le} \bar{y} b < 0$ \RA%
      lehetetlen
    + ezek szerint $y = \frac{\bar{y}}{\lambda}$ létezik, $y A = c$,
      $y \ge 0$, $yb < t$ \qed
+ \emph{\aref{thm:linprog:dual:dual}.~tétel bizonyítása:}
  + (1): már beláttuk a ``3 kalickás tétel''
  (\ref{thm:linprog:farkas:3kalicka}.~tétel) (1)~\LRA~(3) eseténél
    + $cx \le yb$ miatt $cx$ felülről, $yb$ alulról korlátos
  + (2), primál állítás: legyen $t = \sup \{ cx : Ax \le b \}$
    + indir.~tfh.~$\nexists x : Ax \le b, (t \ge )\,cx \ge t$ \RA alkalmazzuk
    \aref{lem:linprog:dual:t}.~lemmát
    + $\exists y : yA = c, y \ge 0, yb < t$ \RA $y$ a duális egy
      megoldása
    +  $t = \sup \{ cx : Ax \le b \} \le yb < t$ \RA ellentmondás \RA $\exists$ primál
      megoldás, hogy $cx \ge t$
    + így a szuprémum egyben maximum kell legyen
  + (2), duális állítás: $\min \{ yb : yA = c, y \ge 0 \} = - \max \{
    -by : A^\T y \le c, -A^\T y \le -c, -y \le 0 \}$
    + alkalmazzuk a primál állítást a duálisra, mint primál feladatra
  + (3): indir.~tfh.~$\exists t : \max \{ cx : Ax \le b \} < t < \min
    \{ yb : yA = c, y \ge 0 \}$
    + a \aref{lem:linprog:dual:t}.~lemma szerint $\exists y : yA = c,
      y \ge 0, yb < t$
    + $t < \min \{ yb : yA = c, y \ge 0 \} < yb < t$ \RA ellentmondás
    \qed
+ \thm \label{thm:linprog:dual:dual}a lineáris programozás
  dualitástétele, ekvivalens alak
  + ha $\max \{ cx : Ax \le b, x \ge 0 \}$ (primál) program megoldható és
    felülről korlátos, akkor
    + (1)~$\min \{ yb : yA \ge c, y \ge 0 \}$ (duális) program megoldható és
      alulról korlátos
    + (2)~a primál programnak $\exists$ maximuma és a duális programnak
      $\exists$ minimuma
    + (3)~$\max \{ cx : Ax \le b, x \ge 0 \} = \min \{ yb : yA \ge c,
      y \ge 0 \}$
  + \proof $\max \{ cx : (A | {-I}) x \le (b | 0) \}$ duálisa $\min \{ y
    (b | 0) : y (A | {-I}) = 0, y \ge 0 \}$
    + legyen $y = (y_1 | y_2)$ \RA $\min \{ y_1 b : y_1 A - y_2 = 0,
      y_1 \ge 0, y_2 \ge 0\}$
    + $y_1 A = y_2 \ge 0$ \RA a duális valóban
      $\min \{ y_1 b : y_1 A \ge 0, y_1 \ge 0 \}$ alakú \qed
+ lineáris programozás bonyolultsága
  + döntési probléma
    + \DataIn $A$ mátrix, $b$ és $c$ vektorok, $t \in \RR$
    + \DataOut van-e olyan $x$ vektor, melyre $Ax \le b$, $cx \ge t$
  + NP-beli \ra tanú a feltételeket kielégítő $x$
  + co-NP-beli \ra a dualitástétel szerint tanú egy $yA = c$, $y \ge
    0$, $(cx \le )\,yb < t$ vektor
  + 1947 (Dantzig): \emph{szimplex-módszer}
    + nem polinomiális futásidejű, de a gyakorlatban gyors
  + 1979 (Hacsijan): \emph{ellipszoid-módszer}
    + polinomiális futásidejű \RA bizonyítja, hogy a feladat P-ben van
    + gyakorlatban a szimplex sokkal hatékonyabb
  + 1984 (\emph{Karmarkar}): polinomiális és a gyakorlatban is használható módszer

\tetel{Egészértékű programozás: a feladat bonyolultsága, korlátozás és
  szétválasztás (Branch and Bound). Totálisan unimoduláris mátrix
  fogalma, példák. Egészértékű programozás totálisan unimoduláris
  együtthatómátrixszal (biz.~nélkül).}

+ \dfn \emph{egészértékű programozás alapfeladata:} $\max_{x} \{ cx : Ax \le b, x
  \text{ egész} \}$ (IP)
  + duálisa $\min_{y} \{ yb : yA = c, y \ge 0, y \text{ egész} \}$ (DIP)
  + $\max_{\IP} \le \max_{\LP} = \min_{\DLP} \le \max_{\DIP}$ \RA állhat
    $<$ is, nincs általános dualitástétel
+ egészértékű programozás bonyolultsága
  + döntési probléma
    + \DataIn $A$ mátrix, $b$ és $c$ vektorok, $t \in \RR$
    + \DataOut van-e olyan $x$ egészértékű vektor, melyre $Ax \le b$, $cx \ge t$
  + NP-beli \ra tanú a feltételeket kielégítő $x$
  + dualitástétel hiányában a co-NP beliség nem látható be
  + \thm az egészértékű lineáris programozás NP-teljes
    + \proof azt már láttuk, hogy IP $\in$ NP
    + adunk egy MAXFTLN $\prec$ IP Karp-redukciót (MAXFTLN NP-teljes)
    + a $G = (V, E)$ gráf $\forall$ $v_i$ csúcsához vegyünk fel egy
      $x_i \in \ZZ$, $-x_i \le 0$, $x_i \le 1$ változót
      + ha $v_i \in F$ a független ponthalmaz eleme, $x_i = 1$,
        egyébként $x_i = 0$
    + $\forall e = \{ v_i, v_j \} \in E$ élre vegyük még fel az $x_i +
      x_j \le 1$ feltételt
    + a célfüggvény $\sum_{v_i \in V} x_i = \lvert F \rvert$ \RA $c =
      (1, 1, \ldots, 1)$
    + $x$ maximumhely \LRA $F \subseteq E$ a $G$ maximális független
      ponthalmaza
      + $F$ független, mert $x_i + x_j \le 1$ \RA $\{ v_i,
        v_j \} \in E$ legfeljebb egyik vége lehet $F$-beli
      + ha $F' \subseteq E$ független, $\lvert F' \rvert > \lvert F
        \rvert$ \RA $x'$ megoldás, $cx' > cx$
      + $x$ nem lehet maximumhely \RA ellentmondás \qed
    + hasonló 3-SAT $\prec$ IP redukciót is lehet adni (0-1
      változók, termek \ra feltételek)
+ \alg Branch and Bound $\max \{ cx : Ax \le b, f \le x \le g; f, g, x
  \text{ egész} \}$ problémára
  + (IP) feladat szétvágása (IP)$'$ és (IP)$''$ feladatokra
    + választunk egy $x_j$ \emph{elágazási változót} és $f_j \le t <
      g_j$ közbülső értéket
    + az új problémák (IP)$'$: (IP) $\cup\;g'_j \coloneqq t$ és
      (IP)$''$: (IP) $\cup\;f''_j \coloneqq t$
  + részproblémák $\mathcal{L} = \{ \text{(IP$^{(i)}$)} = (f^{(i)},
    g^{(i)}, w^{(i)}) : i = 1, 2, \ldots\}$ listáját tartjuk karban,
    ahol $\max \text{(IP$^{(i)}$)} \le w^{(i)}$
  + $x^*$ az eddig megtalált legjobb megoldás, $z^* = c x^*$ a hozzá
    tartozó függvényérték
  + 0.~lépés: $\mathcal{L} \gets \{ (f, g, \infty) \}$, $z^* \gets
    -\infty$, $x^*$ nem definiált
  + 1.~lépés: ha $\mathcal{L} = \emptyset$ \RA STOP, egyébként
    válasszunk egy (IP$^{(i)}$)-t és töröljük $\mathcal{L}$-ből
  + 2.~lépés: ha $w^{(i)} \le z^*$ \RA nem lehet jobb megoldás, mint
    az eddigi, GOTO 1.
  + 3.~lépés: oldjuk meg az (IP$^{(i)}$)-nek megfelelő (LP$^{(i)}$)
    relaxált LP feladatot
    + ha $\nexists$ megoldás \RA GOTO 1., egyébként legyen
      $x^{(i)}$ a megoldás és $c x^{(i)} = z^{(i)}$
  + 4.~lépés
    + (4\rlap{a}\phantom{b})~ha $z^{(i)} \le z^*$ \RA GOTO 1.,
      $f^{(i)} \le x \le g^{(i)}$-ben már nincs jobb megoldás
    + (4b)~ha $z^{(i)} > z^*$ és $x^{(i)}$ egész vektor \RA $x^* \gets
      x^{(i)}$, $z^* \gets z^{(i)}$, GOTO 1.
    + (4\rlap{c}\phantom{b})~ha $z^{(i)} > z^*$, de $x^{(i)}$ nem
      egész vektor
     + vágjuk két részre az (IP$^{(i)}$) problémát valamely $x_j$
       elágazási változó mentén
     + $\mathcal{L} \gets \mathcal{L} \cup \{(f^{(i)\prime},
       g^{(i)\prime}, z^{(i)}), (f^{(i)\prime\prime},
       g^{(i)\prime\prime}, z^{(i)})\}$, GOTO 1.
  + \thm a B\&B véges sok lépésben leáll és megtalálja (IP) optimumát
    + \proof $f$ és $g$ egész \RA véges sok részprobléma van \RA véges
      sok lépés
    + indir.~tfh.~az eljárás leállt, de $z^* < z_0$, ahol $z_0 = \max
      \text{(IP)}$
    + az algoritmus futása közben mindig volt olyan (IP$^{(i)}$) $\in
      \mathcal{L}$, hogy $z_0 = \max \text{(IP$^{(i)}$)}$
      + kezdetben ez maga (IP)
      + $z_0 \le w^{(i)} \le z^* < z_0$ ellentmondás \RA (IP$^{(i)}$)
        mindig eljut a 4.~lépésig
      + (4\rlap{a}\phantom{b})~nem teljesülhet, mert $z_0 \le z^{(i)}
        \le z^* < z_0$ ellentmondás
      + (4b) után $z^* = z^{(i)} = z_0$ lesz \RA $z_0 = z^* < z_0$
        ellentmondás
      + (4\rlap{c}\phantom{b})-ben (IP$^{(i)}$) vágása után $z_0 = \max
        \text{(IP$^{(i)}$)}{}'$ vagy $z_0 = \max \text{(IP$^{(i)}$)}{}''$
    + $\mathcal{L}$ sosem lesz üres \RA az algoritmus nem áll le \RA
      ellentmondás \qed
  + \emph{branch and bound fa:} (IP) a gyökér, (IP$^{(i)}$) gyerekei
    (IP$^{(i)}$)$'$ és (IP$^{(i)}$)$''$
  + heurisztika (IP$^{(i)}$) választására
    + LIFO, ha az (IP$^{(i)}$) a 3.~lépésben utoljára vizsgált probléma fia
      + inkrementális megoldás \emph{duál szimplex módszerrel}
    + egyébként válasszuk azt a problémát, amire $w^{(i)}$ maximális
  + heurisztika az $x_j$ elágazási változó és $t$ választására
    + $x_j$ az változó, aminek az $\bigl\{ x^{(i)}_j \bigr\}$ törtrésze
      legközelebb van $\frac{1}{2}$-hez, $t \gets \bigl\lfloor
      x^{(i)}_j \bigr\rfloor$
  + a gyakorlatban még a heurisztikákkal sem mindig alkalmas nagy
    problémákhoz
+ \dfn $B \in \RR^{k \times k}$ az $A$ mátrix \emph{négyzetes
  részmátrixa}, ha $A$ tetszőleges $k$ sorának és $k$ oszlopának
  kereszteződései határozzák meg
+ \dfn $A$ \emph{totálisan unimoduláris}, ha $\forall$ négyzetes
  részmátrixára $\det = 0, 1 \text{ vagy } {-1}$
  + \corr ha $A$ TU \RA $A$ minden eleme $0$, $1$ vagy $-1$ ($1 \times
    1$-es részmátrix)
+ \lemma \label{lem:linprog:ip:tu}egy mátrix totálisan unimoduláris marad, ha
  + (1)~egy sorát vagy oszlopát $(-1)$-gyel szorozzuk
  + (2)~egységvektort veszünk hozzá új sorként vagy oszlopként
  + (3)~egyik sorát (ill.~oszlopát) új sorként (ill.~oszlopként) új
    példányban hozzávesszük
  + (4)~transzponáljuk
  + \proof csak azoknak a részmátrixoknak változik a $\det$-a,
    amiket a módosítás érint
    + (1)~sor vagy oszlop $(-1)$-gyel szorzása \RA $\det$ $(-1)$-gyel
      szorzása \RA $0$, $1$, $-1$
    + (2)~alkalmazzuk a kifejtési tételt az új sor/oszlop szerint
      + csak egy nemnulla együtthatójú aldetermináns van
      + az új részmátrix determinánsa megegyezik egy régébbiével \RA%
        $0$, $1$ vagy $-1$
    + (3)~két azonos sor/oszlop \RA $\det = 0$
    + (4)~$\det B^\T = \det B$ \RA a determináns nem változik \qed
+ \example \label{ex:linprog:ip:directed}$\vec{G} = (V, \vec{E})$ irányított gráf $A \in \RR^{\lvert
  V \rvert \times \lvert E \rvert}$ illeszkedési
  mátrixsza TU
  + \proof a $B \in \RR^{k \times k}$ részmátrixra $k$ szerinti teljes
    indukcióval
    + $k = 1$-re nyilván $\det B = 0, 1$ vagy $-1$, mert csak ez lehet
      a mátrix eleme
    + ha $B$-ben van csupa $0$ oszlop \RA $\det B = 0$
    + ha $B$-ben van egyetlen $1$-est vagy $-1$-est tartalmazó oszlop
      + fejtsük ki a determinánst eszerint az oszlop szerint
      + az nem $0$ együtthatós $B' \in \RR^{(k - 1) \times (k
        - 1)}$ aldet.-ra $\det B'$ jó az indukció  szerint
      + a kifejtési tétel szerint $\det B = \pm \det B'$
    + ha $B$ $\forall$ oszlopában van $1$ és $-1$ is \RA $(1, 1,
      \ldots, 1) B = 0$ \RA $\det B = 0$ \qed
+ \example \label{ex:linprog:ip:bipartite}$G = (F, L; E)$ páros irányítatlan gráf $A$ illeszkedési
  mátrixsza TU
  + \proof irányítsuk $G$ éleit $F \to L$ irányba \RA $\vec{G}$
    + a $\vec{G}$ irányított gráf $\vec{A}$ illeszkedési mátrixsza TU
      \aref{ex:linprog:ip:directed}.~példa szerint
    + szorozzuk meg az $\vec{A}$ mátrix $L$-hez tartozó sorait $(-1)$-gyel
    + az így kapott mátrix épp $A$, és \aref{lem:linprog:ip:tu}.~lemma
      miatt szintén TU \qed
+ \thm \label{ex:linprog:ip:iplp}ha $A$ TU mátrix, $b$ egész vektor, $c$ valós vektor
  + és $\max \{ cx : Ax \le b \}$ (LP) megoldható és a maximuma véges
  + \RA $\max \{ cx : Ax \le b, x \text{ egész} \}$ (IP) megoldható és
    a maximuma véges
  + és $\max \{ cx : Ax \le b \} = A \max \{ cx : Ax \le b, x \text{
    egész} \}$
  + \noproof

\tetel{A lineáris és egészértékű programozás alkalmazása páros
  gráfokra és intervallumrendszerekre: Egerváry tétele,
  intervallumrendszerek egyenletes színezése.}

+ \thm (Egerváry Jenő tétele) legyen $G = (F, L; E)$ páros gráf, $w\colon E
  \to \RR$ súlyfüggvény \RA a maximális összsúlyú párosítás összsúlya
  $\min \sum_{v \in F \cup L} c(v)$, ahol a minimum
  a nemnegatív $c\colon F \cup L \to \RR_{\ge 0}$ függvényeken értendő, melyekre
  $\forall e = \{x, y\} \in E: c(x) + c(y) \ge w(e)$
  + \proof legyen $G$ illeszkedési mátrixsza $B \in
    \RR^{\lvert F \cup L \rvert \times \lvert E \rvert}$
  + a $B x \le (1, 1, \ldots, 1)^\T$, $x \ge 0$, $x$ egész rendszer
    minden megoldása $0$-$1$ értékű
    + az $1$ komponensek $G$ egy független élhalmazát (párosítását)
      határozzák meg
  + $\max \{ wx : Bx \le (1, 1, \ldots, 1)^\T, x \ge 0, x \text{
    egyész} \}$ megoldása maximális súlyú párosítás
  + $B$ TU mátrix (\ref{ex:linprog:ip:bipartite}.~példa) \RA
    $\max_{\IP} = \max_{\LP}$ (\ref{ex:linprog:ip:iplp}.~tétel)
  + $\max \{ wx : Bx \le (1, 1, \ldots, 1)^\T, x \ge 0 \} = \min \{ y
    (1, 1, \ldots, 1)^\T : yB \ge w, y \ge 0 \}$
  + az $y$ duális megoldás $\forall$ csúcshoz egy $c(v_i) = y_i$
    címkét rendel
    + $y B \ge w$ \RA $\forall \{x, y\} \in E: c(x) + c(y) = w(\{x,
    y\})$ \qed
+ \dfn $G = (\mathcal{I}, E)$ \emph{intervallumgráf}, ha $\mathcal{I}
  = \{I_1, I_2, \ldots, I_m \}$ intervallumok rendszere és $\{ I_i, I_j
  \} \in E$ \LRA $I_i \cap I_j \ne \emptyset$
  + ált.~megsz. nélkült feltehető, hogy $\forall I \in \mathcal{I}$ az $[1, n]$
  egész végpontú, zárt részintervalluma
+ $A(\mathcal{I}) \in \RR^{n \times m}$, ahol $\mathcal{I}$ $[1, n]$ egész
  végpontú intervallumrendszer, $a_{i, j} = \text{\footnotesize$\begin{cases}
      1, &\text{ha $i \in I_j$,} \\
      0, &\text{ha $i \notin I_j$}
    \end{cases}$}$
  + \lemma az így definiált $A(\mathcal{I})$ mátrix TU
  + \proof teljes indukcióval $A$ egyeseinek száma szerint
    + ha $A$-ban $0$ db egyes van \RA $\det A = 0$
    + $\forall$ oszlopban egy darabig $0$-k, utána $1$-esek, majd
      megint $0$-k vannak
    + ha $\exists$ két oszlop, ahol ugyanott van a legfelső egyes
      + a több egyes tartalmazóból a másikat kivonva csökken az
        egyesek száma
      + ez nem változtatja a $\det$-t \RA ind.~feltevés szerint
        $\det = 0, 1$ vagy $-1$
    + ha $\exists$ csupa $0$ oszlop \RA $\det = 0$
    + egyébként minden oszlopban máshol van a legfelső egyes
      + sor- és oszlopcserékkel alsó háromszögmátrixszá alakítható \RA%
      $\det = \pm 1$ \qed
+ \thm \label{thm:liprog:app:interval}az $[1, n]$ egész végpontú, zárt
  $I_1, I_2, \ldots, I_m$
  részintevallumai $\forall k \in \ZZ^{+}$-re megszínezhetőek $k$
  színnel úgy, hogy ha az $i$-t $d_i$ db intervallum tartalmazza,
  akkor ezek között minden színből vagy $\bigl\lfloor \frac{d_i}{k}
  \bigr\rfloor$ vagy $\bigl\lceil \frac{d_i}{k} \bigr\rceil$ van
  + \proof válasszunk ki néhány intervallumot úgy, hogy $\forall i$-re
    $\bigl\lfloor \frac{d_i}{k} \bigr\rfloor$ vagy $\bigl\lceil
    \frac{d_i}{k} \bigr\rceil$ kiválasztott intervallum tartalmazza
    $i$-t
    + ezt az eljárást ismételhetjük $k - 1$-re, $k - 1$-re, \ldots, $1$-re
    + így épp $k$ színnel színezhetőek az intervallumok a megadott
      feltételnek megfelelően
  + legyen $A = A(\mathcal{I})$,
    $\bigl\lfloor \frac{d}{k} \bigr\rfloor$ $i$.~komponense
    $\bigl\lfloor \frac{d_i}{k} \bigr\rfloor$, $\bigl\lceil \frac{d}{k}
    \bigr\rceil$ hasonló
  + $\bigl\lfloor \frac{d}{k} \bigr\rfloor \le A x \le
    \bigl\lceil \frac{d}{k} \bigr\rceil$, $0 \le x \le (1, 1, \ldots,
    1)^\T$ megoldható, pl.~$x = \bigl( \frac{1}{k}, \frac{1}{k},
    \ldots, \frac{1}{k} \bigr)$ megoldás
    + \aref{ex:linprog:ip:iplp}.~tétel miatt $\exists$ egészértékű
      ($0$-$1$) megoldás \RA ez épp egy jó kiválasztás \qed
+ \dfn a $G$ gráf \emph{perfekt}, ha $\chi(F) = \omega(F)$ $\forall F
  \subseteq G$ feszített részgráfjára
  + $\chi$ a kromatikus szám, $\omega$ a klikkszám
+ \corr minden intervallumgráf perfekt
  + \proof a $G$ intervallumgráf minden feszített részgráfja intervallumgráf
    + elég belátni, hogy $\chi(G) = \omega(G)$
  + alkalmazzuk \aref{thm:liprog:app:interval}.~tételt $k =
    \omega(G)$ választással
    + $\forall i: d_i \le \omega(G)$ \RA $\forall$ klikkben $\forall$ színt
      legfeljebb egyszer használtuk
    + a kapott színezés egy jó színezés \RA $\chi(G) \le k = \omega(G)$
  + mivel minden gráfban $\chi(G) \ge \omega(G)$ \RA $\chi(G) =
    \omega(G)$ \qed

\tetel{A lineáris és egészértékű programozás alkalmazása hálózati
  folyamproblémákra: a maximális folyam, a minimális költségű folyam
  és a többtermékes folyam feladatai, ezek hatékony megoldhatósága a
  tört-, illetve egészértékű esetben.}

+ legyen $G = (V, E)$ irányított gráf, $s, t \in V$, $c\colon E \to
  \RR_{+}$ \emph{kapacitásfüggvény}
  + jelölje $x\colon E \to \RR_{\ge 0}$ $v \in V$-be belépő éleken felvett
    összegét $\rho_x(v)$
  + jelölje $x$ a $v$-ből kilépő éleken felvett összegét $\delta_x(v)$
  + \dfn $x\colon E \to \RR_{\ge 0}$ \emph{folyam}, ha $\forall v \in
    V - \{ s, t\}: \rho_x(v) = \delta_x(v)$
  + \dfn $x$ folyam \emph{megengedett}, ha $\forall e
    \in E : x(e) \le c(e)$
  + \dfn az $x$ megengedett folyam \emph{értéke} $\delta_x(s) - \rho_x(s) =
    \rho_x(t) - \delta_x(t)$
+ \dfn a $C = (S, T)$ \emph{vágás}, ha $S \cup T = V$, $S \cap T =
  \emptyset$, $s \in S$, $t \in T$
  + \dfn a $C = (S, T)$ vágás \emph{értéke} $m_C = \sum_{(x, y) \in E,
    x \in S, y \in T} c(x, y)$
  + tetszőleges $C$ vágás értéke felső becslés minden folyam nagyságára
+ \lemma \label{lem:linprog:folyam:folyam}ha $x\colon E \to \RR_{\ge
  0}$-ra $\forall v \in V - \{s, t\}: \rho_x(v) \ge \delta_x(v)$ és
  $\rho_x(t) \ge \delta_x(s)$ \RA $x$ folyam
  + \proof az $S = \sum_{v \in V - \{s, t\}} \rho_x(v) - \delta_x(v)$
    összeg $\forall$ tagja nemnegatív \RA $0 \le S$
  + vegyük észre, hogy $S = \delta_x(s) - \rho_x(t)$
    + ha $e = (u, v)$, $u \ne s$, $v \ne t$ \RA $x(e)$ pozitív és
      negatív előjellel is megjelenik $S$-ben
    + ha $u = s$, $v \ne t$ \RA $x(e)$ csak $+$ előjellel jelenik meg
      $\rho_x(v)$-nél
    + ha $u \ne s$, $v =e t$ \RA $x(e)$ csak $-$ előjellel jelenik meg
      $\delta_x(u)$-nál
    + az $e = (s, t)$ él nem jelenik meg $S$-ben, és kiesik
      $\delta_x(s) - \rho_x(t)$-ben
  + $0 \le S = \delta_x(s) - \rho_x(t) \le 0$ \RA $S = 0$
    + ez csak úgy lehet, ha $\forall$ (nemnegatív) tag $= 0$ \RA%
      $\forall v \in V - \{ s, t\}: \rho_x(v) = \delta_x(v)$ \qed
+ \prob maximális értékű folyam
  + \DataIn $G = (V, E)$ irányított gráf, $s, t \in V$, $c\colon E \to
    \RR^{+}$ kapacitásfüggvény
  + \DataOut az $x$ maximális értékű folyam
  + legyen $G$ illeszkedési mátrixsza $B$, ennek $v \in V$-hez tartozó
    sora $b_v$
  + legyen $G^* \coloneqq (V, E^*)$, $E^* \coloneqq E \cup \{(t,
    s)\}$, az illeszkedési mátrixsza $B^* = (B|b^*)$
  + ha $x^* = (x|\mu)$ és $B^* x^* \le 0$
    + $\forall v \in V - \{s, t\} : b_v x \le0$ \RA $\delta_x(v) -
      \rho_x(v) \le 0$ \RA $\delta_x(v) \le \rho_x(v)$
    + $s$-re és $t$-re nézve  $\delta_x(s) - \mu \le 0$ és
      $-\rho_x(t) + \mu \le 0$ \RA $\delta_x(s) \le \mu \le \rho_x(t)$
    + \aref{lem:linprog:folyam:folyam}.~lemma szerint $x$ valóban
      folyam, az értéke $\mu$
  + max.~folyam lineáris programja: $\max \{ (0, 0, \ldots, 0, 1) x^*
    : B^* x^* \le 0, x^* \ge 0, x \le c\}$
    + ekvivalensen $\max \{ \mu : (B^* | \text{``$I$''}) (x, \mu | x)
      \le (0 , 0 | c), x \ge 0, \mu \ge 0 \}$
    + a második ``egységmátrixból'' hiányzik az $e^* = (t, s)$-hez
      tartozó oszlop
  + a duális feladatot a kényelmesen a dualitástétel ekvivalens
    alakjából kapjuk
    + $\forall v \in V$ csúcshoz $\pi(v)$ és $\forall e \in E$ élhez
      $w(e)$ változó
    + $\min \{ \sum_{e} w(e) c(e) : \forall v: \pi(v) \ge 0; \forall
      e = (u, v): w(e) \ge 0, \pi(u) - \pi(v) + w(e) \ge 0; \\
      \phantom{\min\{} \pi(t) - \pi(s) \ge 1\}$
+ \thm \label{thm:linprog:folyam:dlp}a fenti duális minimuma épp a maximális hálózati folyam értéke
  + \proof (\LA) tetszőleges $C = (S, T)$ vágáshoz $\exists \pi,
    w: \sum_{e} w(e) c(e) = m_C$
    + legyen $\pi(v) = \text{\footnotesize$\begin{cases}
        0, &\text{ha $v \in S$,} \\
        1, &\text{ha $v \in T$}
      \end{cases}$}$, $w(u, v) = \text{\footnotesize$\begin{cases}
        0, &\text{ha $u \in S$, $v \in T$} \\
        1, &\text{egyébként}
      \end{cases}$}$
    + ez kielégíti a feltételeket, és $\sum_{e} w(e) c(e)$ tényleg a
      vágás értéke
  + (\RA) a duális feladdat mátrixsza TU ($G$ illeszkedési mátrixsza +
    egységvektorok)
    + a jobb oldal egészértékű \RA $w$ és $\pi$ is lehet
      egészértékű az optimális megoldásban
    + készítünk egy $0$-$1$ megoldást:
      $\pi'(v) \coloneqq \text{\footnotesize$\begin{cases}
        0, &\text{ha $\pi(v) \le \pi(s)$,} \\
        1, &\text{ha $\pi(v) > \pi(s)$}
      \end{cases}$}$, $w'(e) \coloneqq \text{\footnotesize$\begin{cases}
        0, &\text{ha $w(e) = 0$,} \\
        1, &\text{ha $w(e) \ge 1$}
      \end{cases}$}$
      + $\pi'(t) - \pi'(s) \ge 0$, mert $\pi(t) - \pi(s) \ge 0$ miatt
        $\pi(t) > \pi(s)$ teljesült
      + ha $\pi'(u) - \pi'(v) + w(u, v) < 0$ \RA $\pi'(u) = 0$, $\pi'(v)
        = 1$, $w(u, v) = 0$
      + de $\pi(u) \le \pi(s) < \pi(v)$, $w(e) = 0$ \RA%
        $\pi(u) - \pi(v) + w(e) < 0$, ellentmondás
    + $\forall e: w'(e) \le w(e)$ \RA $\sum_e w'(e) c(e) \le \sum_e
      w(e) c(e)$ \RA $(\pi', w')$ tényleg optimális
    + $S = \{v : \pi'(v) = 0\}$, $T = \{v : \pi'(v) = 1\}$ egy vágás,
      $m_C = \sum_e w'(e) c(e)$
    + a minimális kapacitású vágás kapacitása tényleg a duál optimuma
      + a dualitástétel miatt ez a primál optimuma
      + ami épp a maximális folyam nagysága \qed
  + beláttuk, hogy a maximális folyam polinomiális időben
    meghatározható
    + pl.~ellipszoid módszerrel
      + de a gyakorlatban Edmonds`--Karp algoritmusa nem használ LP-t
    + $(B^* | \text{``$I$''})$ TU \RA ha $c$ egész, a maximális
      egészértékű folyam is meghatározható
+ \thm (Ford`--Fulkerson) a maximális hálózati folyam nagysága
  megegyezik a minimális kapacitású vágás kapacitásával
  + \proof \aref{thm:linprog:folyam:dlp}.~tétel alapján triviális \qed
+ \prob minimális költségű folyam
  + \DataIn $G = (V, E)$, $s, t \in V$, $c\colon E \to \RR^+$,
    $k\colon E \to \RR_{\ge 0}$ \emph{költségfüggvény}, $M$
    folyamérték
  + \DataOut $x$ legalább $M$ értékű, minimális $\sum_{e \in E} k(e)
    x(e)$ költségű folyam
  + $\max \{ -kx : B^* x^* \le 0, x^* \ge 0, x \le c, \mu \ge M \}$ LP
    feladat \RA polinomiális időben megoldható
  + ha $c$ és $M$ egész \RA az IP verzió is polinomiális, mert a
    mátrix megint TU
+ \prob többtermékes folyamprobléma
  + \DataIn $G = (V, E)$, $(s_1, t_1), (s_2, t_2), \ldots, (s_k,
    t_k)$, $c\colon E \to \RR^+$
  + \DataOut $x_1, x_2, \ldots, x_2$ folyamok, melyekre $\forall e \in
    E: \sum_{i = 1}^k x_i(e) \le c(e)$ és az összes folyamnagyság
    $\sum_{i = 1}^k \delta_{x_i}(s) - \rho_{x_i}(t)$ maximális
  + legyen $B_i^*$ a $G_i^* = (V, E \cup (t_i, s_i))$ illeszkedési
    mátrixsza, $x_i^* = (x_i, \mu_i)$
  + $\max \bigl\{ \sum_{i = 1}^k \mu_i : \forall_{i = 1}^k B_i^* x_i^*
    \le 0, x_i^* \ge 0; \bigl( \sum_{i = 1}^k x_i \bigr) \le c
    \bigr\}$ \RA polinomiális időben megoldható
  + az egészértékű változat $k \ge 2$ esetén nem TU
    + ha $k = 1$ \RA maximális folyam probléma, egész $c$ esetén
      polinomiális
    + ha $k = 2$ \RA a feladat NP-nehéz (\noproof)


\input{matroid.tex}

\section{Közelítő és ütemezési algoritmusok} 

\tetel{Polinomiális időben megoldható feladat fogalma, példák. Az NP,
  co-NP, NP-nehéz és NP-teljes problémaosztályok definíciója,
  viszonyaik, példák problémákra valamennyi osztályból. NP-nehéz
  feladatok polinomiális speciális esetei: algoritmus a maximális
  független ponthalmaz problémára és az élszínezési problémára páros
  gráfokon. Additív hibával közelítő algoritmusok speciális pont-,
  illetve élszínezési problémákra.}

+ algoritmuselméleti alapfogalmak
  + \dfn \emph{kiszámítási problémá}ról akkor beszélünk, ha egy $I$
    bemenet $f(I)$ függvényét szeretnénk kimenetként megadni
    + \emph{eldöntési probléma}: $f(I) \in \{\textsc{Igen},
      \textsc{Nem}\}$
  + \dfn \emph{optimalizálási probléma} olyan kiszámítási probléma, ahol
    + az $I$ bemenethez $X_I$ a \emph{lehetséges kimenetek} halmaza,
      $c\colon X_I \to \RR$ valós függvény
    + $f(I) = x^* \in X_I$, hogy $c(x^*) = \max \{ c(x) : x \in X_I \}$
      (illetve $c(x^*) = \min \{ c(x) : x \in X_I \}$)
    + a kimenet nemcsak az optimum értéke, hanem maga az $x^*$ optimális
      megoldás
  + \dfn egy $f \colon \Nat \to \Nat$ függvény \emph{polinomális}, ha
    $\exists c_1, c_2 \in \Nat^*: \forall n \in \Nat : f(n) \le c_1
    n^{c_2}$
    + egy algoritmus akkor \emph{polinomiális} ha $\forall$ $n$ méretű
      bemenetre $f(n)$ időben kiszámítja a kimenetet, ahol $f$
      polinomiális
    + egy probléma \emph{polinom időben megoldható}, ha $\exists$ rá
      polinomiális algoritmus
  + \dfn az $A$ probléma \emph{polinom időben visszavezethető} $B$-re
    ($A \cookprec B$), ha $A$ polinom időben megoldható a
    $B$-t megoldó algoritmus szubrutinként hívásával
    + $B$ hivása $O(1)$ lépésnek számit
  + \dfn döntési problémák osztályai
    + P $\coloneqq$ a polinom időben megoldható eldöntési
      problémák osztálya
      + pl.~teljes párosítás páros gráfban, $k$-matroid-partíció
    + NP $\coloneqq$ azon eldöntési problémák osztálya, ahol az
      \textsc{Igen} válaszra létezik polinomiális méretű, polinomiális
      időben ellenőrizhető tanú
      + pl.~Hamilton-kör, $k$-matroid-metszet ($k \ge 2$)
    + co-NP $\coloneqq$ azon eldöntési problémák osztálya, ahol az
      \textsc{Nem} válaszra létezik polinomiális méretű, polinomiális
      időben ellenőrizhető tanú
      + pl.~teljes párosítás páros gráfban (Kőnig tétele, Hall-tétel),
        síkbarajzolhatóság (Kuratowski-tétel), teljes párosítás
        tetszőleges gráfban (Tutte-tétel)
    + egy $B$ probléma \emph{NP-nehéz}, ha $\forall A \in
      \mathrm{NP}: A \cookprec B$
      + pl.~az általános $k$-polimatroid matching NP-nehéz, de nem
        NP-beli
    + egy $B$ probléma \emph{NP-teljes}, ha NP-beli és NP-nehéz
      + pl.~SAT (Cook`--Levin tétel), 3SZÍN, HAM, LÁDAPAKOLÁS
  + P $\subseteq$ NP $\cap$ co-NP, NP-teljes $=$ NP $\cap$ NP-nehéz, P
    $\overset{?}{=}$ NP
+ \prob maximális független ponthalmaz páros gráfban
  + \DataIn $G = (A, B; E)$ páros gráf\qquad\DataOut $F$ maximális
    független ponthalmaz
  + a probléma általános gráfra NP-nehéz, de páros gráfra polinomiális
  + futtasuk le a gráfra a javítóutas algoritmust (polinom idejű)
  + \stmnt ekkor $F = A_1 \cup A_2 \cup B_1 \cup B_3$ maximális
    független ponthalmaz
    + $A_1, B_1 =$ az $A$-, illetve $B$-beli párosítatlan pontok
      halmaza
    + $B_2 =$ az $A_1$-ből alternáló úton elérhető pontok (nincs
      javító út \RA mind van párjuk)
    + $A_1 =$ a $B_2$-beli pontok párjai
    + $A_3 = A - A_1 - A_2$, $B_3 = B - B_1 - B_2$
    + \proof az algoritmus által megtalált $M$ párosítás maximális \RA%
      $\lvert M \rvert = \nu(G)$
      + $\nu(G) =$ maximális párosítás mérete, $\tau(G) =$ min.~lefogó
        ponthalmaz mérete
      + nincs él $A_1 \cup A_2$ és $B_1 \cup B_2$ között
        (\ref{thm:linprog:egervary:magyar}.~tétel) \RA $F$ független
      + $A \cup B - F = A_3 \cup B_2$ lefogó ponthalmaz \RA $\tau(G)
        \le \lvert A_3 \cup B_2 \rvert = \lvert F \rvert = \nu(G)$
      + egy lefogó ponthalmaz egy párosítás $\forall$ élét le kell
        fogja \RA $\nu(G) \le \tau(G)$
      + így $\tau(G) \le \lvert A_3 \cup B_2 \rvert = \lvert F \rvert
        = \nu(G) \le \tau(G)$ \RA $\nu(G) = \tau(G)$
    + indir.~tfh.~$\exists F' \subseteq A \cup B$ független
      ponthalmaz, $\lvert F' \rvert > \lvert F \rvert$
      + ekkor $A \cup B - F'$ lefogó ponthalmaz, de $\lvert A \cup B -
        F' \rvert < \lvert A \cup B - F \rvert = \tau(G)$
      + ez ellentmondás, $F$ maximális kell legyen \qed
+ \prob élszínezés páros gráfokon
  + \DataIn $G = (A, B; E)$ egyszerű páros gráf\qquad\DataOut $G$ élszínezése
    $\Delta(G)$ színnel
  + a probléma általános gráfra NP-nehéz, de páros gráfra polinomiális
  + \thm \label{thm:kozelito:additiv:vizing}(Vizing tétele) ha $G = (V, E)$ egyszerű gráf \RA $\Delta(G)
    \le \chi_e(G) \le \Delta(G) + 1$
    + \noproof
  + \thm (Kőnig tétele) ha $G = (A, B; E)$ egyszerű páros gráf \RA%
    $\chi_e(G) = \Delta(G)$
    + \proof meg fogunk adni egy polinomiális algoritmust a színezésre
  + vegyük a gráf $E = \{ e_1, e_2, \ldots, e_m \}$ éleit sorban, és
    színezzük meg őket egy-egy szabad színnel
    + legyen $e_i = \{u, v\}$, $ u \in A$, $v \in b$ a most
      megszínezendő él
    + $d(u), d(v) \le \Delta(G)$, és illeszkedik rájuk színezetlen él
      ($e_i$)
      + \RA mindkét csúcsnak $\exists$ szabad színe \RA ha $\exists$
      közös szabad szín, $e_i$ kapja meg azt
  + tfh.~az $u$ szabad színe a \emph{piros}, de a $v$-é a \emph{kék}
    + legyen $P_u$ az $u$-ból induló, felváltva kék--piros út (kör nem
      lehet)
    + ekkor $v \notin P_u$, mert akkor a végpontja lenne ($v$-ben a
      kék szabad)
      + de $u \in A$, $v \in B$ \RA $P_u$ páratlan hosszú \RA $P_u$
        utolsó éle kék
    + cseréljük fel $P_u$ mentén a piros és a kék színeket
      + ez a meglevő színezést nem ronthatja el
      + $u$-ban most már a kék szabad, $v$-ben maradt szabad \RA%
        $e_i$ színezhető kékre \qed
+ \dfn egy maximalizálási (minimalizálási) problémát egy algoritmus
  \emph{$C$ additív hibától eltekintve helyesen old meg}, ha $\forall I:
  c(x^*) \ge \max_{x \in X_i} c(x) - C$ ($c(x^*) \le \min_{x \in X_i}
  c(x) + C$)
+ \dfn egy algoritmus \emph{$C$ additív hibával közelítő algoritmus},
  ha adott optimalizálási problémát polinom időben, $C$ additív
  hibától eltekintve helyesen old meg
+ \prob egyszerű gráfok élszínezése
  + \DataIn $G = (V; E)$ egyszerű gráf\qquad\DataOut $G$
    minimális élszínezése
  + Vizing tétele (\ref{thm:kozelito:additiv:vizing}.~tétel) szerint
    $\chi_e(G) \le \Delta(G) + 1$
    + a tétel bizonyítása konstruktív, ad egy $\Delta(G) + 1$ polinom
      időben
    + így a tételben szereplő algoritmus $1$ additív hibával közelítő
+ \prob síkgráfok csúcsszínezése
  + \DataIn $G = (V; E)$ síkgráf\qquad\DataOut $G$
    minimális csúcsszínezése
  + \thm (négyszíntétel) minden síkgráf színezhető $4$ színnel,
    \noproof
    + létezik algoritmikus bizonyítás, ami polinomidőben előállít egy
      $4$-színezést
    + ez egy $2$ additív hibával közelítő algoritmus
  + ha a $4$-színező algoritmus előtt ellenőrizzük, hogy $G$ páros-e
    \RA $2$-színezhető
    + ez szélességi kereséssel polinomidőben megtehető

\tetel{A Hamilton-kör probléma visszavezetése a leghosszabb kör
  probléma additív közelítésére. $k$-approximációs algoritmus fogalma,
  példák: két-két algoritmus a minimális lefogó ponthalmaz keresésére
  és a maximális páros részgráf keresésére. Minimális levelű, illetve
  maximális belső csúcsú feszítőfa keresése. Approximációs algoritmus
  az utóbbi feladatra (biz.~nélkül).}

+ \prob leghosszabb kör (LHK)
  + \DataIn $G = (V, E)$ gráf\qquad\DataOut $G$ egy leghosszabb köre
  + \thm ha P $\ne$ NP \RA LHK-re nincs $C$ additív hibával
    közelítő algoritmus
    + \proof megmutatjuk, hogy HAM $\prec$ LHK $C$-additív közelítése
    + indir.~tfh.~$\exists$ $C$-additív hibájú közelítés LHK-ra
    + $G' \coloneqq$ osszuk fel a $G$ gráf $\forall$ élét $C$ új
      ponttal \RA egy él helyett $C + 1$ új él
    + $G'$ $k (C + 1)$ hosszú körei $G$ $k$ hosszú köreinek felelnek meg
    + $G'$-ben van $n (C + 1)$ hosszú kör \LRA $G$-ben van $n$ hosszú
      kör \LRA $G$ hamiltoni
    + $G$ hamiltoni \RA a közelítő algoritmus $\ge n (C + 1) - C$
      hosszú $x^*$ kört talál $G'$-ben
      + de $G'$ köreinek hossza osztható $(C + 1)$-gyel \RA $x^*$
        csak a Hamilton-kör lehet \qed
+ \dfn egy maximalizálás (minimalizálási) problémát egy algoritmus
  \emph{$k$ multiplikatív hibától eltekintve helyesen old meg}, ha
  $\forall I : c(x^*) \ge \frac{1}{k} \max_{x \in X_I} c(x)$
  ($c(x^*) \le k \min_{x \in X_I} c(x)$)
+ \dfn egy algoritmus \emph{$k$-approximációs algoritmus}, ha
  polinomiális és $k$ multiplikatív hibától eltekintve helyesen oldja
  meg az adott problémát
+ \prob minimális lefogó ponthalmaz (MLP)
  + \DataIn $G = (V, E)$\qquad\DataOut egy $X$ minimális lefogó ponthalmaz
    $G$-ben
  + MLP NP-nehéz, mert a maximális független halmaz (MFP) is az, és
    MFP $\prec$ MFP
    + $X$ minimális független ponthalmaz \LRA $V - X$ maximális lefogó
      ponthalmaz
    + az MFP nem $k$-approximálható (\noproof)
  + \alg $2$-közelítő algoritmus maximális párosítással
    + 1.~lépés: keressük meg $G$ egy $M$ maximális párosítását
      + páros gráfban: magyar modszer, általános gráfban: Edmonds
        algoritmusa
    + 2.~lépés $X \coloneqq$ az $M$-beli élek végpontjai
    + \thm \label{thm:kozelito:multi:lefogo1}az így meghatározott $X$-re $\lvert X \rvert \le 2 \tau(G)$
      + \proof egy lefogó ponthalmaz $M$ $\forall$ élét le kell fogja
        \RA $\lvert M \rvert = \nu(G) \le \tau(G)$
      + $\lvert F \rvert = 2 \lvert M \rvert = 2 \nu(G) \le 2 \tau(G)$
        \qed
  + \alg $2$-közelítő algoritmus tovább nem bővíthető párosítással
    + 0.~lépés: $M \coloneqq \emptyset$
    + 1.~lépés: ha van olyan $e \in E$, hogy $M \cup \{e\}$ párosítás
      \RA $M \gets M \cup \{e\}$, GOTO 1.
      + egyébként $M$ egy tovább nem bővíthető párosítás
    + 2.~lépés: $X \coloneqq$ az $M$-beli élek végpontjai
    + \thm az így meghatározott $X$-re $\lvert X \rvert \le 2 \tau(G)$
      + \proof \aref{thm:kozelito:multi:lefogo1}.~tételben nem
        használtuk ki, hogy $M$ maximális
      + így továbbra is $\lvert F \rvert = 2 \lvert M \rvert \le 2
        \nu(G) \le 2 \tau(G)$ \qed
  + éles példa: $m$ darab diszjunkt $1$ hosszú út
    + optimális megoldás: mindegyik élnek csak az egyik végét vesszük
      bele \RA $m$ csúcs
    + közelítő megoldás: mindegyik él mindkét vége \RA $2m$ csúcs
+ \prob maximális páros részgráf (MPR) \RA NP-nehéz (\noproof)
    + \DataIn $G = (V, E)$ gráf
    + $A, B: A \cup B = G, A \cap B = \emptyset$, az $A$ és $B$ között
      menő élek száma maximális
    + \alg \label{alg:kozelito:multi:paros1}$2$-approximációs algoritmus 1.
      + 1.~lépés: osszuk ketté a csúcshalmazat $A$-ra és $B$-re
        tetszőlegesen
      + $d_{\mathrlap{s}\phantom{m}}(v) \coloneqq$ $v$-ből a saját
        csoportjába menő élek száma
      + $d_m(v) \coloneqq$ $v$-ből a másik csoportba menő élek száma
      + 2.~lépés: keressünk egy olyan $v$ csúcsot, amire $d_s(v) >
        d_m(v)$
        + ha van ilyen $v$, helyezzük át a másik csoportba; GOTO 2.
      + 3.~lépés: $(A, B)$ egy $2$-approximáció az MPR problémára
      + \thm \label{thm:kozelito:multi:paros1}az így meghatározott $(A, B)$ valóban $2$-approximáció
        + \proof $\forall$ lépésben az $A$ és $B$ között haladó élek
          száma nő \RA $\le \lvert E \rvert$ lépés
        + $\forall v \in V : d_m(v) \ge d_s(v)$ \RA $\frac{1}{2}
          \sum_{v \in V} d_m(v) \ge \frac{1}{2} \sum_{v \in V} d_s(v)$
        + $\frac{1}{2} \sum_{v \in V} d_m(v)$ az $A$ és $B$ között
          menő élek száma
        + $\frac{1}{2} \sum_{v \in V} d_m(v) = \frac{1}{4} \sum_{v
          \in V} d_m(v) + d_m(v) \ge \frac{1}{4} \sum_{v \in V}
          d_m(v) + d_s(v) = \frac{1}{2} \lvert E \rvert \ge
          \frac{1}{2} \lvert E_{\max} \rvert$,
          ahol $E_{\max} \subseteq E$ a maximális páros részgráf
          éleinek halmaza \qed
    + \alg $2$-approximációs algoritmus 2.
      + 1.~lépés $A \coloneqq \emptyset$, $B \coloneqq \emptyset$
      + 2.~lépés $G$ pontjait valamilyen sorrendben véve helyezzük el
        azoka a halmazokba, ahol \emph{kevesebb} szomszédjuk van
      + \thm az így meghatározott $(A, B)$ valóban $2$-approximáció
        + továbbra is igaz az, hogy $\forall v \in V : d_m(v) \ge
        d_s(v)$ \RA lásd \aref{thm:kozelito:multi:paros1}.~tételt \qed
      + jobb közelítéshez (ha szerencsénk van) még futtathatjuk
        \aref{alg:kozelito:multi:paros1}.~algoritmust
+ \prob minimális levelű feszítőfa (MLF)
  + \DataIn $G = (V, E)$ összefüggő gráf
  + \DataOut $F$ minimális számú $1$ fokú csúcsot tartalmazó feszítőfa
    $G$-ben
  + \thm MLF NP-nehéz
    + \proof HAMÚT $\prec$ MLF
      + a Hamilton-út $2$ levelű feszítőfa, ennél kevesebb levelű
        pedig nem létezhet \qed
    + MLF-re nem létezik $k$-approximációs algoritmus (\noproof)
+ \prob maximális belső csúcsú feszítőfa (MBF)
  + \DataIn $G = (V, E)$ összefüggő gráf
  + \DataOut $F$ maximális számú \emph{nem} $1$ fokú csúcsot
    tartalmazó feszítőfa $G$-ben
  + MBF ekvivalens MLF-fel, ezért szintén NP-nehéz, de már
    $k$-approximálható
  + \alg $2$-appriximációs algorimus MBF-re
    + ILST = Independent Leaves Spanning Tree
    + 0.~lépés: legyen $F$ egy tetszőleges feszítőfa $G$-ben
    + 1.~lépés: ha $F$ Hamilton-út, vagy a levelei független halmazt
      alkotnak \RA STOP
    + 2.~lépés: legyen $a$ és $b$ $F$ két levele, ahol $\{a, b\} \in E$
      (szomszédosak $G$-ben)
      + keressük meg a $P = a \leadsto b$ (egyértelmű) utat $F$-ben
      + $F$ nem Hamilton-út \RA $P$-nek van $F$-ben $\ge 3$.~fokú
        csúcsa \RA az egyik legyen $v$
    + 3.~lépés: legyen $w : \{v, w\} \in B$, $F \gets F - \{\{v, w\}\}
      \cup \{\{a, b\}\}$, GOTO 1.
      + az így kapott $F$ továbbra is feszítőfa, a levelek száma
      csökkent
    + \thm a fenti algoritmus polinomiális, $F$ egy $2$-approximáció
      MBF-re
      + \noproof
  + \alg $2$-appriximációs algorimus MBF-re mélységi kereséssel
    + 1.~lépés: legyen $F$ a $G$ mélységi feszítőfája $r \in E$-ből
      indítva
    + 2.~lépés: ha $F$ Hamilton-út, vagy $d(r) > 1$ \RA STOP
      + ha $d(r) = 1$, de $r$ nem szomszédos $F$ másik levelével \RA
      STOP
    + 3.~lépés: legyen $a$ olyan levele $F$-nek, mellyel $r$
    szomszédos
      + keressük meg a $P = r \leadsto a$ utat $F$-ben (egyértelmű)
      + $v$ az $a$-hoz legközelebbi $\ge 3$.~fokú csúcs $P$-ben
      + $w$ a $v$-vel szomszédos csúcs a $v \leadsto a$ úton $P$-ben
        (lehet, hogy $w = a$)
    + 4.~lépés: $F \gets F - \{\{v, w\}\} \cup \{\{r, a\}\}$
    + \thm a fenti algoritmus $O(\lvert E \rvert)$ idejű, $F$ egy $2$-approximáció
      MBF-re
      + \noproof   

\tetel{A minimális lefogó ponthalmaz probléma visszavezetése a
  halmazfedési feladatra, a halmazfedési feladat közelítése, éles
  példa. Közelítő algoritmus a Steiner-fa problémára, éles példa.}

\tetel{A Hamilton-kör probléma visszavezetése az általános utazóügynök
  probléma $k$-app\-ro\-xi\-má\-ci\-ós megoldására. Közelítő
  algoritmusok a metrikus utazóügynök problémára, Christofides
  algoritmusa.}

\tetel{Teljesen polinomiális approximációs séma fogalma. A részösszeg
  probléma, bonyolultsága. Teljesen polinomiális approximációs séma a
  részösszeg problémára.}

\tetel{Ütemezési feladatok típusai. Az
  $1 | \textrm{prec} | C_{\textrm{max}}$ és az $1 \| \Sigma C_j$
  feladat. Approximációs algoritmusok a $P \| C_{\textrm{max}}$
  feladatra: listás ütemezés tetszőleges sorrendben, éles példa
  tetszőleges számú gép esetére; listás ütemezés LPT sorrendben
  (biz.~nélkül), éles példa tetszőleges számú gép
  esetére. Approximációs algoritmus a
  $P | \textrm{prec} |C_{\textrm{max}}$ feladatra (biz.~nélkül),
  példák: az LPT sorrend, illetve a leghosszabb út szerinti ütemezés
  sem jobb, mint $(2 - \frac{m}{1})$-approximáció. A
  $P | \textrm{prec}, p_i = 1 | C_{\textrm{max}}$ feladat, Hu
  algoritmusa (biz.~nélkül).}

\section{Megbízható hálózatok tervezése}

\tetel{Globális és lokális élösszefüggőség és élösszefüggőségi szám
  fogalma, Menger irányítatlan gráfokra és élösszefüggőségre vonatkozó
  két tétele (biz.~nélkül). $\lambda(G)$ meghatározása folyamok
  segítségével négyzetes és lineáris számú folyamkereséssel.}

\tetel{$\lambda(G)$ meghatározása összehúzások segítségével,\\Mader
  tétele, Nagamochi és Ibaraki algoritmusa.}

\tetel{Minimális méretű $2$-élösszefüggő részgráfok keresése.\\A probléma
  NP-nehézsége, Khuller`--Vishkin algoritmus (biz.~nélkül).}

\section{Hálózatelméleti alkalmazások}

\tetel{Kirchhoff tételei a klasszikus villamos hálózatok analízisére.}

\tetel{Kirchhoff eredményeinek általánosítása transzformátorokat vagy
  girátorokat is tartalmazó hálózatokra (biz.~nélkül). Algoritmusok a
  feltételek ellenőrzésére.}

\tetel{Kirchhoff eredményeinek általánosítása: szükséges feltétel
  tetszőleges lineáris sok-ka\-pu\-kat is tartalmazó hálózatok egyértelmű
  megoldhatóságára. Villamos hálózatok duálisa.}

\section{Statikai alkalmazások}

\tetel{Rúdszerkezetek, merevségi mátrix, merevség, egyszerű rácsos
  tartók,\\Cremona`--Maxwell diagramok.}

\tetel{Minimális merev rúdszerkezetek általános helyzetben, Laman
  tétele (biz.~nélkül), Lovász és Yemini tétele.}

\tetel{Síkbeli négyzetrácsok és egyszintes épületek átlós merevítése.}

\end{document}