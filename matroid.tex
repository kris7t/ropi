\section{Matroidok}

\tetel{Matroid definíciója, alapfogalmak (bázis, rang, kör). Példák:
  lineáris matroid (mátrixmatroid), grafikus matroid, uniform
  matroid. A rangfüggvény szubmodularitása.}

+ \dfn az $\MM = (E, \FF)$ pár ($\FF \subseteq 2^E$) \emph{matroid},
  ha
  + (F1)~$\emptyset \in \FF$\qquad(F2)~ha $Y \subseteq X$ és $X \in
    \FF$ \RA $Y \in \FF$
  + (F2)~ha $X, Y \in \FF$ és $\lvert X \rvert > \lvert Y \rvert$ \RA
    $\exists x \in X - Y : Y \cup \{ x \} \in \FF$
+ \dfn az $\MM = (F, \FF)$ matroidban $X \subseteq E$
  \emph{független}, ha $X \in \FF$
  + egyébként $X$ \emph{összefüggő}
+ \dfn az $\MM = (E, \FF)$ matroidban $X \in \FF$ független
  halmaz \emph{bázis}, ha tartalmazásra nézve maximális (tovább már
  nem bővíthető $E$-ben)
+ \dfn az $\MM = (E, \FF)$ matroidban $E \supseteq X \notin \FF$
  összefüggő halmaz \emph{kör}, ha tartalmazásra nézve minimális
  ($\forall$ valódi részhalmaza független)
+ \lemma \label{lem:matroid:def:meret}ha $\MM = (E, \FF)$ matroid; $A
  \subseteq E$; $A \supseteq X_1, X_2 \in \FF$ tovább már nem
  bővítehetőek \RA $\lvert X_1 \rvert = \lvert X_2 \rvert$
  + \proof indir.~tfh.~$\lvert X_1 \rvert < \lvert X_2 \rvert$
    $\xRightarrow{\text{(F3)}}$ $\exists e \in X_2 - X_1 : X_1
    \subset X_1 \cup \{ e \} \in \FF$
  + ez ellentmondás, $X_1$ mégis tovább bővíthető \qed
  + azt kaptuk, hogy tetszőleges $X$ független halmaz kiegészíthető
    maximálissá
+ \corr az $\MM$ matroid $\forall$ bázisa azonos elemszámú
+ \dfn $\MM = (E, \FF)$ matroidban az $X \subseteq E$ halmaz
  \emph{rang}ja $r(X) =$ az $X$-beli maximális független halmaz mérete
  (\aref{lem:matroid:def:meret}.~lemma szerint ez mindig egyértelmű)
+ \example  a $G = (V, E)$ gráf \emph{grafikus matroid}ja $\MM(G) =
  (E, \FF)$, ahol $F \in \FF$ pontosan akkor, ha az $F$ élek erdőt
  alkotnak $G$-ben
+ \example valamely $T$ test felett az $M \in T^{n \times m}$ mátrix
  oszlopvektorai által indukált matroid \emph{mátrixmatroid}, ahol
  a független halmazok a lineárisan független vektorhalmazok
+ $\UU_{n, k} = (E, \FF)$ \emph{uniform matroid}, ahol $\lvert E
  \rvert = n$, $\FF = \{ X \subseteq E : \lvert X \rvert \le k \}$
  + $\UU_{n, n}$ \emph{teljes} (\emph{szabad}) \emph{matroid},
    $\UU_{n, 0}$ \emph{triviális matroid}
+ \thm \label{thm:matroid:def:submod}$r(x)$ \emph{szubmoduláris}, azaz
  $\forall X, Y \subseteq E: r(X) + r(Y) \ge r(X \cap Y) + r(X \cup Y)$
  + \proof legyen $A$ maximális független halmaz $X \cap Y$-ban
    \RA $r(A) = r(X \cap Y)$
  + legyen $B \supseteq A$ maximális független halmaz $X \cup Y$-ban
    \RA $r(B) = r(X \cup Y)$
  + $B \cap (X \cap Y) = A$ \RA $\lvert B \cap X \rvert + \lvert B
    \cap Y \rvert = \lvert B \rvert + \lvert A \rvert$
  + $B \cap X, B \cap Y \subseteq B \in \FF$ függetlenek \RA%
    $\lvert B \cap X \rvert \le r(X)$, $\lvert B \cap Y \rvert \le
    r(Y)$
  + $r(X) + r(Y) \ge \lvert B \rvert + \lvert A \rvert = r(X \cup Y) +
    r(X \cap Y)$ \qed

\tetel{Mohó algoritmus matroidon. Matroid megadása rangfüggvényével,
  bázisaival (biz.~nélkül).  Matroid duálisa, a duális matroid
  rangfüggvénye.}

+ \alg mohó algoritmus
  + \DataIn $\FF \subseteq 2^E$ halmazrendszer, ami kielégíti az
    (F1--2) tulajdonságokat (nem üres, ``leszálló''), $w\colon E \to
    \RR^+$ \emph{súlyfüggvény}
  + \DataOut olyan $X \in \FF$, melyre $\sum_{e \in X} w(e)$ maximális
  + 0.~lépés: legyen $X \coloneqq \emptyset \in \FF$
  + 1.~lépés: létezi-e $ e \in E - X: X \cup \{ e \} \in \FF$
    + $\exists$ \RA $e \coloneqq \argmax \{ w(e') : e' \in E - X, X
    \cup \{ e \} \in \FF$ \}; $\nexists$ \RA STOP
  + 2.~lépés: $X \gets X \cup \{ e \}$, GOTO 1.
+ \thm a mohó algoritmus pontosan akkor ad helyes eredmény $\forall w$
  súlfüggvénnyel az $(E, \FF)$ halmazrendszeren, ha az (F1--2) mellett
  (F3)-at is kielégíti ($(E, \FF)$ matroid)
  + \proof (\LA) matroidon működik a mohó algoritmus
    + indir.~tfh.~$X = \{a_1, a_2, \ldots, a_n\}$ a mohó algoritmus
      kimenete
    + de $\exists Y = \{b_1, b_2, \ldots, b_m\} \in \FF$, $w(Y) >
      w(X)$
    + ált.~megsz.~nélkül tfh.~{\footnotesize$\begin{array}{@{}*{10}{c@{\;}}c@{}}
        w(a_1) & \ge & w(a_2) & \ge\cdots\ge & w(a_{k - 1}) & \ge & w(a_k) & \ge & w(a_{k + 1})
        & \ge\cdots\ge & w(a_n) \\
        \vertle & & \vertle & & \vertle & & \vertgt & & & & \\
        w(b_1) & \ge & w(b_2) & \ge\cdots\ge & w(b_{k - 1}) & \ge & w(b_k) & \ge & w(b_{k + 1})
        & \ge\cdots\ge & w(b_m)
      \end{array}$}
    + $A \coloneqq \{ a_1, a_2, \ldots, a_{k - 1} \} \in \FF$, $B \coloneqq \{
      b_1, b_2, \ldots, b_k \} \in \FF$, $\lvert A \rvert \le \lvert B \rvert$
      + $\xRightarrow{\text{(F3)}}$ $\exists b^* \in B : A \cup \{ b^* \}
        \in \FF$, és a megadott sorrend miatt $w(b^*) \ge w(b_k) > a_k$
      + a mohó algoritmus a $k$. lépésben nem választhatta $a_k$-t,
        mert $b^*$ jobb nála, ellentmondás
  + (\RA) ha (F3) nem teljesül, $\exists$ olyan $w$, hogy a mohó
    algoritmus rossz eredményt ad
    + $\exists X, Y \in \FF : \lvert X \rvert > \lvert Y \rvert$, de
      $\nexists x \in X - Y : Y \cup \{ x \} \in \FF$;
      legyen $w(e) \coloneqq \text{\footnotesize$\begin{cases}
        1, &\text{ha $e \in Y$,} \\
        1 - \epsilon &\text{ha $e \in X - Y$,} \\
        0 & \text{egyébként}
      \end{cases}$}$
    + a mohó algoritmus beválasztja $Y$ elemeit, majd nem tud több nem
      $0$ súlyú elemet beválasztani \RA $w(Y) = \lvert Y \rvert$ a
      maximum
    + de $w(X) = \lvert X \cap Y \rvert + (1 - \epsilon) \lvert X - Y
      \rvert = \lvert X \rvert - \epsilon \lvert X - Y \rvert
      > w(Y) = \lvert Y \rvert$, ha $0 < \epsilon < \frac{\lvert X
        \lvert - \rvert Y \lvert}{\lvert X - Y \rvert}$
    + a mohó algoritmus nem adott optimális megoldást $w$-re! \qed
+ \thm ha $r$ az $\MM = (E, \FF)$ matroid rangfüggvénye, akkor
  + (R1)~$r(\emptyset) = 0$\qquad(R2)~$\forall X \subseteq E: r(x) \le
    \lvert X \rvert$\qquad(R3)~ha $Y \subseteq X$ \RA $r(Y) \le r(X)$
  + (R4)~ $\forall X, Y \subseteq E: r(X) + r(Y) \ge r(X \cap Y) + r(X
    \cup Y)$, azaz $r$ \emph{szubmoduláris}
  + ha $r$ kielégíti az (R1--4) feltételeket \RA az
    $\FF = \{ X \subseteq E : r(X) = \lvert X \rvert \}$
    rangfüggvénye
% + \proof
%   + (R1): (F1) miatt $\emptyset \in \FF$, így $\emptyset$
%     maximális független részhalmaza épp önmaga
%   + (R2): $\FF \ni Y \subseteq X$ \RA $\lvert Y \rvert \le \lvert X
%     \rvert$, így a maximális $Y$ sem lehet $X$-nél nagyobb
%   + (R3): $\forall Z: \FF \ni Z \subseteq Y : Z \subseteq X$, így
%     $Y$-ban nem lehet $X$ max.~független halmazánál nagyobb független
%     halmaz
%   + (R4): lásd \aref{thm:matroid:def:submod}.~tételt
% + megfordítás:
%   + (F1): $r(\emptyset) = \lvert \emptyset \rvert = 0$
%     \RA $\emptyset \in \FF$
%   + (F2): legyen $r(X) = \lvert X \rvert$, $Y \subseteq X$
%     + (R4) \RA $r(Y) + r(X - Y) \ge r(Y \cap (X - Y)) + r(Y
%       \cup (X - Y)) = r(\emptyset) + r(X) = \lvert X \rvert$
%     + (R2) \RA $\lvert X \rvert = \lvert Y \rvert + \lvert X - Y \rvert \ge
%       r(Y) + r(X - Y) \ge \lvert X \rvert$
%     + $\underbrace{r(Y)}_{\le \lvert Y \rvert} + \underbrace{r(X -
%       Y)}_{\le \lvert X - Y \rvert} = \lvert X \rvert$ csak úgy lehetséges, ha
%       $r(Y) = \lvert Y \rvert$, $r(X - Y) = \lvert X - Y \rvert$
%     + vagyis $Y$ és $X - Y \subseteq X$ függetlenek
  + \noproof
+ \thm ha $\BB$ az $\MM = (E, \FF)$ matroid bázisainak halmaza, akkor
  + (B1)~$\BB \ne \emptyset$\qquad(B2)~$\forall X_1, X_2 \in \BB:
    \lvert X_1 \rvert = \lvert X_2 \rvert$
  + (B3)~$X_1, X_2 \in \BB$, $e_1 \in X_1$ \RA $\exists e_2 \in X_2: X
  - \{ e_1 \} \cup \{ e_2 \} \in \BB$
  + ha $\BB$ kielégíti a (B1--3) feltételeket \RA az
    $\FF = \{ X \subseteq E : X \subseteq B \in \BB \}$
    bázisainak halmaza
  + \noproof
+ \dfn az $\MM = (E, \BB)$ matroid \emph{duálisa} $\MM^* = (E,
  \BB^*)$
  + $\BB^* = \{ X \subseteq E : E - X \in \BB \}$ \LRA $\FF^* = \{ X
  \subseteq E : E - X \text{ tartalmazza $\MM$ egy bázisát}\}$
  + $(\MM^*)^* = \MM$ \qquad \example $\UU_{n, k}$ duálisa $(\UU_{n,
    k})^* = \UU_{n, n - k}$
+ \thm az $\MM = (E, r)$ duálisának rangfüggvénye $r^*(X) = \lvert X
  \rvert - r(E) + r(E - X)$
  + \proof $r^*(X) = \text{$X$-beli max.~független halmaz mérete} =
    \max \{ \lvert X \cap B^* \rvert : B^* \in \BB^* \}$
  + $\lvert X \cap B^* \rvert = \lvert X \cap (E - B) \rvert = \lvert
    X - B \rvert = \lvert X \rvert - \lvert X \cap B \rvert$, ahol $E
    - B^* = B \in \BB$
  + $r^*(X) = \max (\lvert X \rvert - \lvert X \cap B \rvert) = \lvert X \rvert -
    \min \lvert X \cap B \rvert$ = $\lvert X \rvert - \lvert B \rvert +
    \max \lvert (E - X) \cap B \rvert$
  + $\max \lvert (E - X) \cap B \rvert = r(E - X)$ \RA $r^*(X) = \lvert X
    \rvert - r(E) + r(E - X)$ \qed

\tetel{Elhagyás és összehúzás. Matroidok direkt összege,
  összefüggősége. $T$ test felett reprezentálható matroid duálisának $T$
  feletti reprezentálhatósága.}

+ \dfn az $\MM = (E, \FF)$ matroidból $X$ \emph{elhagyás}ával az $\MM
  \setminus X = (E - X, \FF')$ matroidot kapjuk, ahol $\FF' \coloneqq \{ F \in
  \FF : F \subseteq E - X \}$
  + grafikus matroidoknál ez nem más, mint élek elhagyása a gráfból
  + $\MM \setminus X$ rangfüggvénye $r \lvert_{E - X}$
+ \dfn az $\MM = (E, r)$ matroidból $X$ \emph{összehúzás}ával az $\MM
  / X = (E - X, r')$ matroidot kapjuk, ahol $r'(Y) = r(X \cup Y) -
  r(X)$
  + \lemma az $(E - X, r')$ pár valóban matroid
  + \proof (R1): $r'(\emptyset) = r(X \cup \emptyset) - r(X) = 0$
    + (R2): $r'(Y) = r(X \cup Y) - r(X) \overset{\text{$r$ (R4)}}{\le}
      (r(X) + r(Y) - r(X \cap Y)) - r(X) \le r(Y) \le \lvert Y \rvert$
    + (R3): $Z \subseteq Y$ \RA%
      $r'(Z) = r(X \cup Z) - r(X) \overset{\!\!\text{$X \cup Z \subseteq
          Z \cup Y$, $r$ (R3)}\!\!}{\le} r(X \cup Y) - r(X) = r'(Y)$
    + (R4): $r'(Y) + r'(Z) = r(X \cup Y) + r(X \cup Z) - 2 r(Z)
      \overset{\text{$r$ (R4)}}{\ge} \\
      \phantom{(R4):}\ge r((X \cup Y) \cap (X \cup Z)) + r((X \cup Y)
      \cup (X \cup Z)) - 2 r(X) =\\
      \phantom{(R4):}= r(X \cup (Y \cap Z)) + r(X \cup (Y \cup Z)) - 2
      r(X) = \\
      \phantom{(R4):}= r'(Y \cap Z) + r'(Y \cup Z)$ \qed
+ \thm ha $\MM = (E, \FF)$ matroid, $X \subseteq E$ \RA $(\MM / X)^* =
  \MM^* \setminus X$, $(\MM \setminus X)^* = \MM^* / X$
  + \proof $\MM^* \setminus X = ((\MM^* \setminus X)^*)^* = ((\MM^*)^*
    / X)^* = (\MM / X)^*$
    + ezért elég az $(\MM \setminus X)^* = \MM^* / X$ egyenlőséget
      belátni
  + legyen $(\MM \setminus X)^* = (E - X, r_1)$, $\MM^* / X = (E - X,
    r_2)$
    + $r_1(Y) = \lvert Y \rvert - r(E - X) + r((E - X) - Y)$
    + $r_2(Y) = r^*(X \cup Y) - r^*(X) = \lvert X \cup Y \rvert -
      r(E) + r(E - (X \cup Y)) - \lvert X \rvert + r(E) - r(E - X)$
      + $Y \subseteq E - X$ \RA $X \cap Y = \emptyset$ \RA $\lvert X
        \cup Y \rvert - \lvert X \rvert = \lvert Y \rvert$
      + $(E - X) - Y = E - (X \cup Y)$ \RA $r_2(Y) = \lvert Y \rvert
        - r(E - X) + r((E - X) - Y)$
    + $r_1 = r_2$, ezért a két matroid is megegyezik \qed
+ \dfn az $\MM$ matroidból összehúzások és törlések sorozatával $\MM$
  \emph{minor}a áll elő
+ \lemma \label{lem:matroid:muvelet:komm}$\MM \setminus A_1 / B_1
  \setminus A_2 / B_2 \setminus \cdots \setminus  A_k / B_k = \MM
  \setminus \bigcup_{i = 0}^k A_i / \bigcup_{i = 0}^k B_i$
   + \noproof
+ \thm az $\MM$ matroid $\forall$ minora előáll $\MM \setminus A / B$
  alakban
  + \proof a \ref{lem:matroid:muvelet:komm}.~lemma alapján triviális
    \qed
+ \dfn az $\MM_1 = (E_1, \FF_1)$, $\MM_2 = (E_2, \FF_2)$, $E_1 \cap
  E_2 = \emptyset$ matroidok \emph{direkt összeg}e $\NN = \MM_1 \oplus
  \MM_2 = (E, \FF)$, $E = E_1 \cup E_2$, $\FF = \{ X \subseteq E : X
  \cap E_1 \in \FF_1, X \cap E_2 \in \FF_2 \}$
+ \dfn az $\MM$ matroid \emph{összefüggő}, ha nem áll elő két matroid
  direkt összegeként
+ \dfn az $\MM$ matroid \emph{reprezentálható}
  (\emph{kooridnátázható}) az $T$ test felett, ha izomorf egy $T$
  feletti mátrixmatroiddal
  + ha $\MM$-et $A$ koordinátázza, és $B \in \RR^{r \times r}$
    négyzetes nemszinguláris mátrix, akkor $BA$ is $\MM$-et
    koordinátázza
+ \thm \label{thm:matroid:muvelet:dualrep}ha $\MM$ reprezentálható a $T$ test felett, akkor $\MM^*$ is
  + \proof legyen $r \coloneqq r(\MM) = r(A)$, ahol $A \in \RR^{r
    \times n}$ az $\MM$-et reprezentáló mátrix
    + Gauss-eliminációval hozzuk $\MM$ mátrixszát $(I_r | A_0)$
      alakúra
  + vizsgáljuk az $A' = (-A_0 | I_{n - r})$ által reprezentált
    $\MM(A')$ matroidot
    + $A$ és $A'$ oszlopai természetes módon megfeleltehetőek egymásnak
    + legyen $B$ egy bázis $\MM$-ben
      + ált.~megsz.~tfh.~$B$ az $I_r$ utolsó $t$ oszlopából és $A_0$
        első $r - t$ oszlopából áll
      + ekkor $E - B$ a $-A_0^\T$ első $r - t$ és az $E_{n - r}$
        utolsó $n - 2r - t$ oszlopából áll
      + a $B$ elemeihez tartozó $r \times (t + (r - t))$-s részmátrix $A$-ban: $A_1 =$
        \scalebox{0.5}{$\left(\begin{array}{@{}c*{2}{@{\,}c}|*{3}{@{\;\;}c}@{}}
          \phantom{C} & \phantom{C} & \phantom{C} & \phantom{C} & \phantom{C} & \phantom{C} \\
          & 0 & & & \text{\raisebox{-0.5ex}{\smash{\clap{\huge$C$}}}} & \\
          & & & & & \\\hline
          1 & & & & & \\[-1ex]
          & \ddots & & & & \\[-1ex]
          & & 1 & & &
        \end{array}\right)$}
      + az $E - B$ elemeihez tartozó részmátrix $A'$-ben: $A'_1 =$
        \scalebox{0.5}{$\left(\begin{array}{@{}c*{2}{@{\;\;}c}|c|*{3}{@{\,}c}@{}}
          \phantom{C} & \phantom{C} & \phantom{C} & \cdots & \phantom{C} & \phantom{C} & \phantom{C} \\
          & \text{\raisebox{-0.5ex}{\smash{\clap{\huge$-C^\T$}}}} & & \cdots & & 0 & \\
          & & & \cdots & & & \\\hline
          & & & \cdots & 1 & & \\[-1ex]
          & & & \cdots & & \ddots & \\[-1ex]
          & & & \cdots & & & 1
        \end{array}\right)$}
    + $r = r(B)$ \LRA $0 \ne \rvert \det A_1 \lvert = \rvert \det C \lvert = \rvert
      \det -C^T \lvert = \rvert \det A_2 \lvert$ \LRA $n - r = r(E -
      B)$
  + $r(\MM(A')) \le n - r = r(E - B)$ \RA $r(\MM(A')) = n - r$, $E -
    B$ valóban bázis $\MM(A')$-ben
    + bázis komplementere bázis (és ezt fordítva is elmondhatjuk)
    + $\BB(A') = \{ E - B : B \in \BB(A) \}$ \RA $\MM(A') = \MM^*$ \qed

\tetel{Grafikus, kografikus, reguláris, bináris és lineáris matroid
  fogalma, ezek kapcsolata (ebből bizonyítás csak a grafikus és
  reguláris matroidok közötti kapcsolatra), példák. Fano-matroid,
  példa nemlineáris matroidra. Bináris, reguláris és grafikus
  matroidok jellemzése tiltott minorokkal: Tutte tételei
  (biz.~nélkül).}

+ \dfn az $\MM = (E, \FF)$ matroid \emph{grafikus}, ha előáll valamely
  $G = (V, E)$ gráf $\MM(G)$ körmatroidjaként, ahol $\FF$ elemei a
  $G$-beli erdők
+ \dfn az $\MM$ matroid \emph{kografikus}, ha egy grafikus matroid
  $[\MM(G)]^*$ duálisa
  + $\FF$ elemei nem tartalmaznak $G$-beli \emph{vágást} (vágásmatroid)
  + $X \in \FF$ \LRA $(V(G), E(G) - X)$ ugyanannyi összefüggő
    komponensből áll, mint $G$
+ \thm egy $\MM(G)$ grafikus matroid pontosan akkor kografikus, ha $G$
  síkbarajzolható
  + ekkor $[\MM(G)]^* = \MM(G^*)$, a matroid duális épp a gráfelméleti
    duális körmatroidja
  + \noproof
+ \dfn az $\MM$ matroid \emph{lineáris}, ha reprezentálható valamely
  $T$ test felett
+ \dfn az $\MM$ matroid \emph{bináris}, ha reprezentálható a $\GF$
  kételemű test felett
  + minden bináris matroid lineáris a $\GF$ test felett
+ \dfn az $\MM$ matroid \emph{reguláris}, ha minden $T$ test felett
  reprezentálható
  + minden reguláris matroid bináris (és lineáris) $T = \GF$
    választással
+ matroidok halmazainak kapcsolatai
  + grafikus, kografikus $\subset$ reguláris $\subset$ bináris
    $\subset$ lineáris $\subset$ matroidok
  + grafikus $\cap$ kografikus $\ne \emptyset$
+ \thm minden $\MM(G)$ grafikus matroid reguláris
  + \proof legyen $T$ test, $V(G) = \{ x_1, x_2, \ldots, x_n \}$ és
    irányitsuk $G$ éleit tetszőlegesen
  + rendeljük az $x_i$ csúcshoz a $v_i$ egységvektort és az $e = (x_i,
    x_j)$ élhez a $w_e = v_i - v_j$
    + ez épp $G$ egy irányításának illeszkedési mátrixsza
  + $X \subseteq E$ tartalmaz kört \RA $W = \{ w_e : e \in X \}$
    lineárisan összefüggő
    + legyen $C \subseteq X$ egy kör $X$-ben valamilyen irányítással
    + ha $e \in C$ párhuzamos $C$-vel, $\alpha_e \coloneqq 1$; ha
      ellentétes irányú, $\alpha_e \coloneqq -1$
    + ekkor $\sum_{e \in C} \alpha_e w_e = 0$ $W$ elemeinek nemtriviális
      $0$ lineáris kombinációja
  + $X \subseteq E$ tartalmaz kört \LA $W = \{ w_e : e \in X \}$
    lineárisan összefüggő
    + legyen $\sum_{w_e \in W} \beta_e w_e = 0$ egy nemtriviális $0$
      linearis kombináció
    + $Y \coloneqq \{ e \in X : \beta_e \neq 0 \}$ és $V' \coloneqq
      \text{az $Y$-beli élek végpontjainak halmaza}$
    + $\sum_{e \in Y} \beta_e w_e = 0$ \RA $\forall v \in V'$-re
      legalább $2$ $Y$-beli él illeszkedik
    + a $(V', Y)$ részgráf $\forall$ pontja legalább $2$.-fokú \RA%
      $(V', Y) \subseteq G$ tartalmaz kört \qed
+ \thm minden $[\MM(G)]^*$ kografikus matroid reguláris
  + \proof \aref{thm:matroid:muvelet:dualrep} tétel szerint ha $\MM$
    reprezentálható $T$ felett, akkor $\MM^*$ is
  + $\MM(G)$ minden $T$ felett reprezentálható \RA $[\MM(G)]^*$ is
    \qed
+ \dfn a $T$ test \emph{karakterisztiká}ja az a min.~$k$, melyre
  $\forall x \in T : \overbrace{x + x + \ldots + x}^{\text{$k$ db}} =
  0$
  + ha valamely $x \ne 0$-ra $\overbrace{x + \ldots + x}^{\text{$k$ db}} =
    0$, akkor ez $\forall y \in T$-re is teljesül, mert $\overbrace{
    y + y + \ldots + y}^{\text{$k$ db}} = \frac{y}{x} (\overbrace{x +
    x + \ldots + x}^{\text{$k$ db}}) = \frac{y}{x} \cdot 0$
+ \example $\FF_7 =$\kern-2ex\begin{tikzpicture}[baseline=0]
    \node (tri) [draw,regular polygon,regular polygon sides=3,inner
    sep=0pt,minimum width=2.3cm] {};
    \node (e) [vertex,label={90:$e$}] at (tri.corner 1) {};
    \node (f) [vertex,label={150:$f$}] at ($(tri.corner 1)!0.5!(tri.corner 2)$) {};
    \node (a) [vertex,label={-120:$a$}] at (tri.corner 2) {};
    \node (b) [vertex,label={-90:$b$}] at ($(tri.corner 2)!0.5!(tri.corner 3)$) {};
    \node (c) [vertex,label={-60:$c$}] at (tri.corner 3) {};
    \node (d) [vertex,label={30:$d$}] at ($(tri.corner 3)!0.5!(tri.corner 1)$) {};
    \node (g) [vertex,label={[xshift=-4pt]90:$g$}] at (tri) {};
    \draw (a) -- (d) (e) -- (b) (c) -- (f);
    \node [draw] at (g) [circle through=(b)] {};
  \end{tikzpicture}\kern-2ex{ Fano-matroid,}
  $\FF_7^-=$\kern-2ex\begin{tikzpicture}[baseline=0]
    \node (tri) [draw,regular polygon,regular polygon sides=3,inner
    sep=0pt,minimum width=2.3cm] {};
    \node (e) [vertex,label={90:$e$}] at (tri.corner 1) {};
    \node (f) [vertex,label={150:$f$}] at ($(tri.corner 1)!0.5!(tri.corner 2)$) {};
    \node (a) [vertex,label={-120:$a$}] at (tri.corner 2) {};
    \node (b) [vertex,label={-90:$b$}] at ($(tri.corner 2)!0.5!(tri.corner 3)$) {};
    \node (c) [vertex,label={-60:$c$}] at (tri.corner 3) {};
    \node (d) [vertex,label={30:$d$}] at ($(tri.corner 3)!0.5!(tri.corner 1)$) {};
    \node (g) [vertex,label={[xshift=-4pt]90:$g$}] at (tri) {};
    \draw (a) -- (d) (e) -- (b) (c) -- (f);
  \end{tikzpicture}\kern-2ex{anti-Fano-matroid}
  + $\forall$ 4 elemű halmaz összefüggő, a 3 eleműek közül az egy
    egyenesre / körre esők
  + $\FF_7$ pontosan a $k = 2$ karakterisztikájú testek felett
    koordinátázható
  + $\FF_7^-$ pontosan a $k \ne 2$ karakterisztikájú
    testek felett koordinátázható
  + \proof $r(\FF_7) = r(\FF_7^-) = 3$ \RA feltehető: $a
    \mapsto (1, 0, 0)$, $c \mapsto (0, 1, 0)$, $e \mapsto (0, 0, 1)$
    + ekkor $b \mapsto (x, y, 0)$, $d \mapsto (0, z, u)$, $f \mapsto
      (v, 0, w)$, $g \mapsto (q, r, s)$
    + $\{ a, d, g \} \notin \FF$ \RA {\footnotesize$\begin{vmatrix}
        1 & 0 & q \\
        0 & z & r \\
        0 & u & s
      \end{vmatrix}$} $= 0$ \RA $sz = ru$; hasonlóan $sv = qw$, $rx =
    qy$
    + $\{ b, d, f \} \notin \FF(\FF_7)$, de $\in \FF(\FF_7^-)$ \RA%
      {\footnotesize$\begin{vmatrix}
        x & 0 & v \\
        y & z & 0 \\
        0 & u & w
      \end{vmatrix}$} $= 0$ $\FF_7$-ben, de $= 0$ $\FF_7^-$-ban
      + $wxz + uvy = 0$ $\FF_7$-ben, de $\ne$ $\FF_7^-$-ban
      + de $\frac{sz}{u} x = \frac{sv}{w} y$ \RA $swxz = suvy$ \RA%
        $wxz = uvy$
      + $wxz + wxz = 0$ $\FF_7$-ben pontosan akkor teljesül, ha $T$
        karakterisztikája $= 2$
      + $wxz + wxz \ne 0$ $\FF_7^-$-ban pontosan akkor teljesül, ha $T$
        karakterisztikája $\ne 2$ \qed
  + \corr az $\FF_7 \oplus \FF_7^-$ matroid nem lineáris
    + $T$ karakteriszikája egyszerre kellene $2$ és nem $2$ legyen
+ \example példák matroidokra
  \par
  {\centering\begin{tabular}{>{\footnotesize}l|l||>{\footnotesize}l|l}
    nem lineáris & $\FF_7 \oplus \FF_7^-$ &
    lineáris, de nem bináris & $\UU_{4, 2}, \FF_7^-$ \\\hline
    bináris, de nem reguláris & $\FF_7$ &
    reguláris, de nem grafikus & $\MM(K_5) \oplus [\MM(K_5)]^*$ \\\hline
    grafikus, de nem kografikus & $\MM(K_5)$ &
    kografikus, de nem grafikus & $[\MM(K_5)]^*$ \\\hline
    grafikus és kografikus & \multicolumn{3}{l}{$\MM(G)$, $G$ síkbarajzolható}
  \end{tabular}\par}
  + $K_5$ helyett tetszőleges nem síkbarajzolható gráf (pl.~$K_{3,3}$) állhat
+ \thm Tutte tételei
  + $\MM$ bináris \LRA nem minorja az $\UU_{4, 2}$
    + $\UU_{4, 2}$ rangja $2$, de $2$ dimenziós bináris vektorokból
      nem létezik $4$ db páronként lineárisan független;
      $\UU_{4, 2} = \MM \text{\footnotesize$\begin{pmatrix}
          1 & 0 & 1 & 1 \\
          0 & 1 & 1 & \mathbf{2}
        \end{pmatrix}$}$
  + $\MM$ reguláris \LRA nem minorja az $\UU_{4, 2}$,
  $\FF_7$ és $\FF_7^*$
  + $\MM$ grafikus \LRA nem minorja az $\UU_{4, 2}$,
    $\FF_7$, ${\FF_7^*}$, ${[\MM(K_5)]^*}$ és ${[\MM(K_{3, 3})]^*}$
    + ha $\MM(G)$ egy $G'$ gráf $[\MM(G')]^*$ vágásmatroidja, akkor
      $\MM(G)$ kografikus és $G'$ síkbarajzolható kell legyen
  + \noproof
+ \thm $\UU_{n, k}$ pontosan akkor grafikus, ha $k = 0, 1, n - 1$ vagy
  $n$
  + \proof $\UU_{n, 0}$ egy csúccsal és $n$ hurokkal reprezentálható,
    $\UU_{n, 1}$ $n$ párhuzamos éllel
  + $\UU_{n, n - 1}$ $n$ élű körrel reprezentálható, $\UU_{n, n}$ $n$
    elű erdővel
  + $\UU_{n, k} \setminus \{a\} = \UU_{n - 1, k}$, $\UU_{n, k} / \{a\}
    = \UU_{n - 1, k - 1}$
    + ha $2 \le k \le n - 2$ \RA $\UU_{n, k}$-nak van $\UU_{4, 2}$
      minorja
    + ehhez $n - k - 2$ törlés és $k - 2$ összehúzás kell \qed

\tetel{Matroidok összege. $k$-matroid-metszet probléma, ennek
  bonyolultsága $k \ge 3$ esetén.}

+ \dfn az $\MM_1 = (E, \FF_1), \MM_2 = (E, \FF_2), \ldots, \MM_k = (E,
  \FF_k)$ matroidok \emph{összeg}e az $\NN = \bigvee_{i = 1}^k \MM_i =
  (E, \FF')$ halmazrendszer, ahol
  $\FF' = \{ X_1 \cup X_2 \cup \ldots \cup X_k :  X_i \in \FF_i \; (i = 1, 2,
  \ldots, k) \}$
  + $\bigvee_{i = 1}^k (\MM_i \setminus X) = \bigl( \bigvee_{i = 1}^k
    \MM_i \bigr) \setminus X$, de az $/$ és a $\oplus$ műveletekre ez
    már nem igaz
+ \thm matroidok $\bigvee_{i = 1}^k \MM_i$ összege matroidot alkot
  + \proof $\MM_1 \vee (\MM_2 \vee \MM_3) = (\MM_1 \vee \MM_2) \vee
    \MM_3$ \RA elegendő $\MM_1 \vee \MM_2$-t vizsgálni
  + (F1): $\emptyset = \emptyset \cup \emptyset \in \FF_1 \vee \FF_2$
  + (F2): $X = X_1 \cup X_2$, $X_1 \in \FF_1$, $X_2 \in \FF_2$, $Y
    \subseteq X$
    + ekkor $X_1 \supseteq Y \cap X_2 \in \FF_1$, $X_2
      \supseteq Y \cap X_2 \in \FF_2$, $(Y \cap X_2) \cup (Y \cap X_2) =
      Y \in \FF'$
  + (F3): legyen $X, Y \in \FF'$; $\lvert X \rvert > \lvert Y \rvert$;
    $X = X_1 \cup X_2$; $Y = Y_1 \cup Y_2$; $X_1, X_2 \in \FF_1$; $Y_1,
    Y_2 \in \FF_2$
    + ált.~megsz.~nélkül~tfh.~$\rvert X_1 \cap Y_2 \lvert +
      \lvert X_2 \cap Y_1 \rvert$ minimális, $X_1 \cap X_2 = Y_1 \cap
      Y_2 = \emptyset$
    + ált.~megsz.~nélkül~tfh.~$\lvert X_1 \rvert > \lvert Y_1 \rvert$
      \RA $\exists e \in X_1 - Y_1 : Y_1 \cup \{ e \} \in \FF_1 \subseteq \FF_1$
    + ha $e \notin Y_2$ \RA $e \in X - Y$, $Y \cup \{ e \} = (Y_1 \cup \{e\}) \cup
      Y_2 \in \FF'$
    + ha $e \in Y_2$ \RA $Y = Y_1' \cup Y_2' = (Y_1 \cup \{e \})
      \cap (Y_2 - \{e \})$ is egy jó felbontása $Y$-nak
      + vegyük észre, hogy $X_1 \cap X_2 = \emptyset$, $e \in X_1$ \RA $e \notin X_2$
      + $\rvert X_1 \cap (Y_2 - \{e\}) \lvert + \lvert X_2 \cap (Y_1
        \cup \{e\}) \rvert = \rvert X_1 \cap Y_2 \lvert - 1 + \lvert X_2
        \cap Y_1 \rvert$
      + a metszetek méretének összege nem lehetett minimális,
        ellentmondás \qed
+ \dfn az $\MM_1 = (E, \FF_1), \MM_2 = (E, \FF_2), \ldots, \MM_k = (E,
  \FF_k)$ matroidok \emph{metszet}e az $(E, \FF')$ halmazrendszer, ahol
  $\FF' = \bigcap_{i = 1}^k \FF_i$
  + ez nem mindig matroid, pl.~%
  $\MM\mleft(\begin{tikzpicture}[baseline=-.5ex,font=\footnotesize]
    \node (a) [vertex] at (0,0) {};
    \node (b) [vertex] at (0.8,0) {};
    \node (c) [vertex] at (1.6,0) {};
    \draw (a) edge node [above] {1} (b);
    \draw (b) edge [bend left] node [above] {2} (c);
    \draw (b) edge [bend right] node [below] {3} (c);
  \end{tikzpicture}\mright)$ és
  $\MM\mleft(\begin{tikzpicture}[baseline=-.5ex,font=\footnotesize]
    \node (a) [vertex] at (0,0) {};
    \node (b) [vertex] at (0.8,0) {};
    \node (c) [vertex] at (1.6,0) {};
    \draw (b) edge node [above] {2} (c);
    \draw (a) edge [bend left] node [above] {1} (b);
    \draw (a) edge [bend right] node [below] {3} (b);
  \end{tikzpicture}\mright)$
  metszete nem az, mert $\FF' = \{ \emptyset, \{1\}, \{2\}, \{3\},
    \{2, 3\}\}$ \RA (F3) nem teljesül
+ \prob (súlyozott) $k$-matroid metszet ($k$-MMP)
  + \DataIn $\MM_1 = (E, \FF_1), \MM_2 = (E, \FF_2), \ldots, \MM_k = (E,
    \FF_k)$, $p \in \Nat$ ($w\colon E \to \RR_{\ge 0}$, $W \in \RR_{\ge 0}$)
  + \DataOut $\exists$-e olyan $X \in \bigcap_{i = 1}^k \FF_i$, hogy
    $\lvert X \rvert \ge p$ (illetve $w(X) \ge W$)
+ \thm $k \ge 3$ esetén $k$-MMP NP-nehéz
  + \proof $3$-MPP NP-nehéz \RA $k > 3$, $k$-MMP NP-nehéz
    + adunk egy $3$-MMP $\prec$ $k$-MMP Karp-redukciót ($k > 3$)
    + $(\MM_1, \MM_2, \MM_3) \in \text{$3$-MPP}$ \LRA $(\MM_1, \MM_2,
      \overbrace{\MM_3, \MM_3, \ldots, \MM_3}^{\text{$k - 2$ db}}) \in
      \text{$k$-MPP}$
  + irányított $s$-$t$-HAMÚT $\prec$ $3$-MMP \RA $3$-MPP NP-teljes
    + $\MM_1 = \MM(G)$
    + $\FF(\MM_2) =$ azon élhalmazok, melyekben $\forall$ pont be-foka
      $\le 1$, $d_{\text{be}}(s) = 0$
      + egy partíció-matroidból kapjuk $s$ bemenő éleinek összehúzásával
    + $\FF(\MM_3) =$ azon élhalmazok, melyekben $\forall$ pont
      \rlap{ki-foka}\phantom{be-foka} $\le 1$,
      $d_{\text{\rlap{ki}\phantom{be}}}(\mathrlap{t}\phantom{s}) = 0$
    + $p = \lvert V \rvert - 1$ \RA $X \in \FF_1, \FF_2, \FF_3$;
      $\lvert X \rvert \ge p$ épp $s$-$t$ Hamilton-út \qed

\tetel{A $k$-matroid partíciós probléma, ennek algoritmikus
  megoldása. A $2$-matroid-metszet feladat visszavezetése matroid
  partíciós problémára.}

+ \prob $k$-matroid partíció ($k$-MPP)
  + \DataIn $\MM_1 = (E, \FF_1), \MM_2 = (E, \FF_2), \ldots, \MM_k = (E,
    \FF_k)$
  + \DataOut $E$ független-e $\bigvee_{i = 1}^k \MM_i$-ben? \LRA
    $\bigvee_{i = 1}^k \MM_i \overset{?}{=} (E, 2^E)$
    + \LRA $\overset{?}{\exists} E_1 \in \FF_1, E_2 \in \FF_2, \ldots,
      E_k \in \FF_k : E = E_1 \cup E_2 \cup \ldots \cup E_k$
+ \alg $k$-matroid partíció algoritmus\footnote{Edmonds, J.~és
  D.~R.~Fulkerson (1965). Transversals and Matroid Partition. In:
  \emph{J.~OF~RESEARCH of the Nat.~Bureau of Standards---B.~Math.~and
    Math.~Phys.~Vol.~69B, No.~3}}
  + 0.~lépés: $E_1 \gets \emptyset$, $E_2 \gets \emptyset$, \ldots,
    $E_k \gets \emptyset$
  + 1.~lépés ha $\bigcup_{i = 1}^k E_i = E$ \RA STOP, igaz
  + 2.~lépés van-e $x \in E - \bigcup_{i = 1}^k E_i$, hogy
    $\bigcup_{i = 1}^k E_i \cup \{x\}$-t partícionálható?
    + ha igen \RA legyen $E_1, E_2, \ldots E_k$ az új partícionálás,
      GOTO 1.
    + ha nem \RA STOP, hamis
  + irányított $\vec{G} = (V, \vec{E})$ segédgráf konstrukció a
    2.~lépés elvégzéséhez
    + $V = E \cup P = E \cup \{ p_1, p_2, \ldots, p_k\}$
    + $(a, p_i) \in \vec{E}$ \LRA $a \notin E_i$, de $E_i \cup \{a\}
      \in \FF_i$
    + $(a, b) \in \vec{E}$ \LRA valamely $i$-re $b \in E_i$, $a \notin
      E_i$, $E_i \cup \{a\} \notin \FF_i$, $E_i \cup \{a\} - \{b\} \in
      \FF_i$
  + szélésségi kereséssel keressük meg a legrövidebb utat $E' = E -
    \bigcup_{i = 1}^k E_i$-ből $P$-be
    + van \RA $(a, b)$-kre $b \in E_i$ esetén $E_i \gets
      E_i \cup \{a\} - \{b\}$; $(a, p_i)$-re $E_i \gets E_i
      \cup \{a\}$
      + $\vec{G}$ definíciója miatt még $\forall E_i \in
        \FF_i$; $E'$ mérete $1$-gyel (út első csúcsa) nőtt
    + nincs \RA legyen $X \subset E$ az $E'$-ből irányított úton
      elérhető csúcsok halmaza
      + $r(X) \le \sum_{i = 1}^k r_i(X) < \lvert X \rvert$ \RA $E
        \supset X \notin \bigcup_{i = 1}^k \FF_i$
        $\xRightarrow{\text{(F2)}}$ $E$ nem lehet
        független
  + lépésszám: $O(\lvert E \rvert + \overbrace{c (n +
    k)^2}^{\text{gráf}} + \overbrace{c (n +
    k)^2}^{\text{szélességi}})$
+ \thm az $\MM_1 = (E, \FF_1)$ és $M_2 = (E, \FF_2)$ azonos rangú
  ($r_1(E) = r_2(E)$) matroidoknak pontosan akkor van közös bázisa, ha
  $\MM_1 \vee \MM_2^* = (E, 2^E)$
  + (\RA) ha $B$ közös bázis, akkor $E - X$ bázisa $\MM_2^*$-nak
    + $E = X \cup (E - X)$ független az összegben \RA $\MM_1 \vee
      \MM_2^*$ teljes
  + (\LA) legyen $E = E_1 \cup E_2$, $E_1 \in \FF_1$, $E_2 \in
    \FF_2^*$
    + $\lvert E \rvert \le \lvert E_1 \rvert + \lvert E_2 \rvert =
      r_1(E_1) + r_2^* (E_2) = r_1(E_1) + (\lvert E \rvert - r_2(E) +
      r_2(E - E_2)) \le\\\phantom{\lvert E \rvert}\le r_1(E) + (\lvert E
      \rvert - r_2(E)) = \lvert E \rvert$
    + mindenhol egyenlőség kell álljon, tehát $\lvert E \rvert \le
    \lvert E_1 \rvert + \lvert E_2$ \RA $E_1 = E - E_2$
    + $r_1(E_1) = r_1(E)$, $r_2(E) = r_2(E - E_2) = r_2(E_1)$ \RA%
      $E_1$ közös bázis \qed
+ \corr $2$-MMP visszavezetése $2$-MPP
  + legyen $\MM_1' = (E, \FF_1')$,  $\MM_2' = (E, \FF_2')$, ahol
    $\FF_i' = \{ X \in \FF_i : \lvert X \rvert \le p \}$ a matroidok
    \emph{csonkoltjai}
  + $\forall$ $p$ méretű független halmaz bázis a megfelelő csonkolt
    matroidban
  + $\MM_1' \vee (\MM_2')^* = (E, 2^E)$ \LRA $\MM_1'$-nek és
    $\MM_2'$-nek van közös bázisa
  + ez a bázis a $2$-MMP-ben keresett $p$ méretű közös
  független halmaz $\MM_1$-ben és $\MM_2$-ben

\tetel{$k$-polimatroid rangfüggvény fogalma. A
  $2$-polimatroid-matching probléma, ennek bonyolultsága, Lovász
  tétele (biz.~nélkül).}

+ \dfn $r\colon 2^E \to \Nat$ \emph{$k$-polimatroid rangfüggvény}, ha
  + (R1)~$r(\emptyset) = 0$\qquad(R2$'$)~$\forall e \in E : r(\{e\})
    \le k$\qquad(R3)~$Y \subseteq X \implies r(Y) \le r(X)$
  + (R4)~$r(X) + r(Y) \ge r(X \cap Y) + r(X \cup Y)$
  + a matroid rangfüggvények $k = 1$ választással adódnak
    + $X = \{e_1\} \cup \{e_2\} \cup \ldots \cup \{e_m\}$
      $\xRightarrow{\text{(R4)}}$ $m \le \sum_{i = 1}^m r(\{e_i\}) \le
      r(X)$
  + másik irány: $\forall X \subseteq E : r(X) \le k \lvert X \rvert$
+ \dfn $X \subseteq E$ egy $k$-matching, ha $r(X) = k \lvert X \rvert$
+ \example $G = (V, E)$ gráf, $r(X) =$ az $X \subseteq E$ élek által
  lefedett csúcsok halmaza
  + $r$ egy $2$-polimatroid-rangfüggvény, a $2$-matchingek $G$
    párosításai
+ \prob $k$-polimatroid matching ($k$-PMMP)
  + \DataIn $r\colon 2^E \to \Nat$ $k$-polimatroid rangfüggvény, $p
    \in \Nat$
  + \DataOut létezik-e $X \subseteq E$, hogy $r(X) = k \lvert X
    \rvert$ és $\lvert X \rvert \ge p$
+ \thm a $k$-PMMP $k \ge 3$ esetén NP-nehéz
  + \proof $k$-MMP $\prec$ $k$-PMMP és $k$-MMP NP-nehéz
  + ha $\MM_1, \MM_2, \ldots, \MM_k$ matroidok \RA $r(X) = \sum_{i =
    1}^k r_i(X)$ $k$-polimatroid rangfüggvény
    + a $k$-matchingek pontosan a  független halmazok $\forall \MM_i$-ben
    + $(\MM_1, \MM_2, \ldots, \MM_k; p) \in \text{$k$-MMP}$ \LRA%
      $\bigl(\sum_{i = 1}^k r_i, p\bigr) \in \text{$k$-PMMP}$ \qed
+ \thm ha $r$ mindössze egy $O(1)$ időben kiszámolható orákulummal
  adott \RA\\$2$-PMMP nem oldható meg polinomidőben
  + \proof legyen $\lvert E \rvert = 2n$, $E_0 \subset E$, $\lvert E_0
    \rvert = n$
  + $r_1(X) \coloneqq \text{\footnotesize$\begin{cases}
      2 \lvert X \rvert, &\text{ha $\lvert X \rvert < n$,} \\
      2n - 1, &\text{ha $\lvert X \rvert = n$,} \\
      2n, &\text{ha $\lvert X \rvert > n$,}
    \end{cases}$}$ $r_2(X) \coloneqq \text{\footnotesize$\begin{cases}
      2 \lvert X \rvert, &\text{ha $\lvert X \rvert < n$,} \\
      2n, &\text{ha $X = E_0$,} \\
      2n - 1, &\text{ha $X \ne E_0, \lvert X \rvert = n$,} \\
      2n, &\text{ha $\lvert X \rvert > n$,}
    \end{cases}$}$ $p \coloneqq n$
  + $(r_1, p) \notin \text{$2$-PMMP}$, de $(r_2, p) \in
    \text{$2$-PMMP}$ ($E_0$ egy $p = n$ elemű $k$-matching)
    + ennek eldöntéséhez az összes $n$ elemű részhalmazt meg kell vizsgálni
    + ez $\binom{2n}{n} = O(4^n / \sqrt{\pi n})$ darab $r$ hívást igényel \qed
  + nem használtuk ki, hogy $2$-PMMP NP-nehéz, a bizonyítás P = NP
    esetén is működik
+ \thm (Lovász László tétele)
  + ha az $r$ $2$-polimatroid rangfüggvény
    $r(X) = \operatorname{dim} \bigl\langle \bigcup_{i \in X} \{ a_i, b_i
    \} \bigr\rangle$ alakban adott, ahol $E = \{1, 2, \ldots, n\}$,
    $\{a_i, b_i\}_{i \in I} \subset \RR^k$
    \RA $r$-en $2$-PMMP polinomidőben megoldható
  + az ilyen $r$ mindig $2$-polimatroid rangfüggvény
  + \noproof
